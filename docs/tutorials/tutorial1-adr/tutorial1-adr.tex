% !TEX root = ../tutorials.tex
\chapter{Advection Diffusion Reaction (ADR) Solver}
\label{ADR}

\section*{Introduction}
Welcome to the Advection Diffusion Reaction (ADR) Solver tutorial for the Nektar++ library.
This tutorial is intended to show the main features of the ADR solver in a simple and user-friendly 
format. If you do not have already downloaded and installed Nektar++, please do so by visiting 
\href{http://www.nektar.info}{nektar.info}, where you can also find the user-guide with the instructions 
to install the library. This tutorial requires only the pre- and post-processing tools and the ADR 
solver.



\subsection*{Goals}
After the completion of this tutorial, the user will be familiar with:
\vspace{-0.5cm}
\begin{itemize}
\item the ADR solver;
\item the pre- and post-processing routines;
\item the external mesh generator Gmsh.
\end{itemize}

\subsection*{Resources}
The files necessary to run this tutorial are included in ??.  
Specifically, two files with extension \textsf{.xml} are needed as input to the code: 
\vspace{-0.5cm}
\begin{enumerate}
\item a file containing the mesh: \textsf{.xml};
\item a configuration file describing the options for the particular problem \textsf{.xml} .
\end{enumerate}

\begin{center}
Now that you are ready, lets' start!
\end{center}

\section{Background}
The ADR solver can solve various problems, including the unsteady advection, unsteady diffusion, 
unsteady advection diffusion equation, etc. For a more detailed description of this solver, please 
refer to the User-Guide. 

In this tutorial we focus on the unsteady advection equation
\begin{equation}
\dfrac{\partial u}{\partial t} + \mathbf{V}\cdot\nabla u = 0,
\label{eq:advection}
\end{equation}
where $u$ is the independent variable and $\mathbf{V} = [\text{V}_{x}\; \text{V}_{y}\; \text{V}_{z}]$ 
is the advection velocity. The unsteady advection equation can be solved in one, two and three spatial 
dimensions. We will here consider a two-dimensional problem, so that $\mathbf{V} = [\text{V}_{x}\; \text{V}_{y}]$.

\section{Problem description}
The problem we want to run consists of a two-dimensional Gaussian function travelling in the $x$-direction
at a constant advection velocity. In order to model this problem we can create a computational domain 
(also referred to as mesh or grid) on which we need to apply a two-dimensional Gaussian function 
as initial condition and periodic boundary conditions at the mesh boundaries. 
We can successively setup the other parameters of the problem, such as the time-step, the time-integration 
scheme, the I/O configuration, etc.
We can finally run the solver and post-process the data in order to visualise them.

\section{Pre-processing}


\section{Configuring the input files}
 

\section{Running the solver}


\section{Post-processing}


\section{Results}