\chapter{Command-line Options}

%\begin{lstlisting}
%--verbose
%\end{lstlisting}
\lstinline[style=BashInputStyle]{--verbose}\\
\hangindent=1.5cm
Displays extra info.

\lstinline[style=BashInputStyle]{--version}\\
\hangindent=1.5cm
Displays software version, and source control information if applicable.

\lstinline[style=BashInputStyle]{--help}\\
\hangindent=1.5cm
Displays help information about the available command-line options for the executable.

\lstinline[style=BashInputStyle]{--parameter [key]=[value]}\\
\hangindent=1.5cm
Override a parameter (or define a new one) specified in the XML file.

\lstinline[style=BashInputStyle]{--solverinfo [key]=[value]}\\
\hangindent=1.5cm
Override a solverinfo (or define a new one) specified in the XML file.

\lstinline[style=BashInputStyle]{--shared-filesystem}\\
\hangindent=1.5cm
By default when running in parallel the complete mesh is loaded by all processes, although partitioning is done uniquely on the root process only and communicated to the other processes. Each process then writes out its own partition to the local working directory. This is the most robust approach in accounting for systems where the distributed nodes do not share a common filesystem. In the case that there is a common filesystem, this option forces only the root process to load the complete mesh, perform partitioning and write out the session files for all partitions. This avoids potential memory issues when multiple processes attempt to load the complete mesh on a single node.

\lstinline[style=BashInputStyle]{--npx [int]}\\
\hangindent=1.5cm
When using a fully-Fourier expansion, specifies the number of processes to use in the x-coordinate direction.

\lstinline[style=BashInputStyle]{--npy [int]}\\
\hangindent=1.5cm
\quad When using a fully-Fourier expansion or 3D expansion with two Fourier directions, specifies the number of processes to use in the y-coordinate direction.

\lstinline[style=BashInputStyle]{--npz [int]}\\
\hangindent=1.5cm
When using Fourier expansions, specifies the number of processes to use in the z-coordinate direction.

\lstinline[style=BashInputStyle]{--part-info}\\
\hangindent=1.5cm
Prints detailed information about the generated partitioning, such as number of
elements, number of local degrees of freedom and the number of boundary degrees
of freedom.

\lstinline[style=BashInputStyle]{--part-only [int]}\\
\hangindent=1.5cm
Partition the mesh only into the specified number of partitions, write to file
and exit. This can be used to pre-partition a very large mesh on a single
high-memory node, prior to being executed on a multi-node cluster.

\lstinline[style=BashInputStyle]{--use-metis}\\
\hangindent=1.5cm
Forces the use of METIS for mesh partitioning. If \nekpp{} is compiled with
Scotch support, the default is to use Scotch.

\lstinline[style=BashInputStyle]{--use-scotch}\\
\hangindent=1.5cm
Forces the use of Scotch for mesh partitioning.
