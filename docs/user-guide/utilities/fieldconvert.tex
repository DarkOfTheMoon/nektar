\chapter{FieldConvert}
\label{s:utilities:fieldconvert}
FieldConvert is a utility embedded in \nekpp with the primary
aim of allowing the user to convert the \nekpp output binary files
(.chk and .fld) into a format which can be read by two common
visualisation softwares: Paraview (.vtu format) or Tecplot (.dat
format). FieldConvert also allows the user to manipulate the
\nekpp output binary files by using some additional modules
which can be called with the option \inltt{-m} which stands
for \inltt{m}odule. Note that another flag, \inltt{-r} (which stand for
\inltt{r}ange) allows the user to specify a sub-range of the domain
on which the conversion or manipulation of the \nekpp output binary
files will be performed.

Almost all of the FieldConvert functionalities can be run in parallel if \nekpp
is compiled using MPI (see the installation documentation for additional info on
how to implement \nekpp using MPI). \footnote{Modules that do not have parallel
  support will be specified in the appropriate section.}
%
%
%
\section{Convert .fld / .chk files into Paraview, VisIt or Tecplot format}
\label{s:utilities:fieldconvert:sub:convert}
To convert the \nekpp output binary files (.chk and .fld) into a
format which can be read by two common visualisation softwares:
Paraview (.vtu format), VisIt (.vtu format) or Tecplot (.dat format)
the user can run the following commands:
%
\begin{itemize}
\item Paraview or VisIt (.vtu format)
%
\begin{lstlisting}[style=BashInputStyle]
FieldConvert test.xml test.fld test.vtu
\end{lstlisting}
%
\item Tecplot (.dat format)
%
\begin{lstlisting}[style=BashInputStyle]
FieldConvert test.xml test.fld test.dat
\end{lstlisting}
%
\end{itemize}
%
where \inltt{FieldConvert} is the executable associated to the utility
FieldConvert, \inltt{test.xml} is the session file and \inltt{test.dat},
\inltt{test.vtu} are the desired format outputs, either Tecplot or
Paraview format respectively.
%
\begin{tipbox}
Note that the session file is also supported
in its compressed format \inltt{test.xml.gz}.
\end{tipbox}
%
%
%
\section{Convert field files between XML and HDF5 format}
%
When \nekpp is compiled with HDF5 support, solvers can select the format used
for output of \inltt{.fld} files. FieldConvert can be used to convert between
these formats using an option on the \inltt{.fld} output module. For example, if
\inltt{in.fld} is stored in the default XML format, it can be converted to HDF5
format by issuing the command
%
\begin{lstlisting}[style=BashInputStyle]
FieldConvert in.fld out.fld:fld:format=Hdf5
\end{lstlisting}
% 
\section{Range option \textit{-r}}
The Fieldconvert range option \inltt{-r} allows the user to specify
a sub-range of the mesh (computational domain) by using an
additional flag, \inltt{-r} (which stands for \inltt{r}ange and either
convert or manipulate the \nekpp output binary files.
Taking as an example the conversion of the \nekpp binary files
(.chk or .fld) shown before and wanting to convert just the 2D
sub-range defined by $-2\leq x \leq 3$, $-1\leq y \leq 2$ the
additional flag \inltt{-r} can be used as follows:
%
\begin{itemize}
\item Paraview or VisIt (.vtu format)
%
\begin{lstlisting}[style=BashInputStyle]
FieldConvert -r -2,3,-1,2 test.xml test.fld test.vtu
\end{lstlisting}
%
\item Tecplot (.dat format)
%
\begin{lstlisting}[style=BashInputStyle]
FieldConvert -r 2,3,-1,2 test.xml test.fld test.dat
\end{lstlisting}
%
\end{itemize}
where \inltt{-r} defines the range option of the FieldConvert
utility, the two first numbers define the range in $x$ direction
and the the third and fourth number specify the $y$ range.
A sub-range of a 3D domain can also be specified.
For doing so, a third set of numbers has to be provided
to define the $z$ range.
%
%
%
\section{FieldConvert modules \textit{-m}}
FieldConvert allows the user to manipulate the \nekpp output
binary files (.chk and .fld) by using the flag \inltt{-m} (which
stands for \inltt{m}odule)..
Specifically, FieldConvert has these additional functionalities
%
\begin{enumerate}
\item \inltt{C0Projection}: Computes the C0 projection of a given output file;
\item \inltt{QCriterion}: Computes the Q-Criterion for a given output file;
\item \inltt{addcompositeid}: Adds the composite ID of an element as an additional field;
\item \inltt{addFld}: Sum two .fld files;
\item \inltt{combineAvg}: Combine two \nekpp binary output (.chk or .fld) field file containing averages of fields (and
possibly also Reynolds stresses) into single file;
\item \inltt{concatenate}: Concatenate a \nekpp binary output (.chk or .fld) field file into single file;
\item \inltt{equispacedoutput}: Write data as equi-spaced output using simplices to represent the data for connecting points;
\item \inltt{extract}: Extract a boundary field;
\item \inltt{homplane}: Extract a plane from 3DH1D expansions;
\item \inltt{homstretch}: Stretch a 3DH1D expansion by an integer factor;
\item \inltt{innerproduct}: take the inner product between one or a series of fields with another field (or series of fields). 
\item \inltt{interpfield}: Interpolates one field to another, requires fromxml, fromfld to be defined;
\item \inltt{interppointdatatofld}: Interpolates given discrete data using a finite difference approximation to a fld file given an xml file;
\item \inltt{interppoints}: Interpolates a set of points to another, requires fromfld and fromxml to be defined, a line or plane of points can be defined;
\item \inltt{isocontour}: Extract an isocontour of ``fieldid'' variable and at value ``fieldvalue''. Optionally ``fieldstr'' can be specified for a string defiition or ``smooth'' for smoothing;
\item \inltt{jacobianenergy}: Shows high frequency energy of Jacobian;
\item \inltt{qualitymetric}: Evaluate a quality metric of the underlying mesh to show mesh quality;
\item \inltt{meanmode}: Extract mean mode (plane zero) of 3DH1D expansions;
\item \inltt{pointdatatofld}: Given discrete data at quadrature points
  project them onto an expansion basis and output fld file; 
\item \inltt{printfldnorms}: Print L2 and LInf norms to stdout;
\item \inltt{scalargrad}: Computes scalar gradient field;
\item \inltt{scaleinputfld}: Rescale input field by a constant factor;
\item \inltt{shear}: Computes time-averaged shear stress metrics: TAWSS, OSI, transWSS, TAAFI, TACFI, WSSG;
\item \inltt{surfdistance}: Computes height of a prismatic boundary layer mesh and projects onto the surface (for e.g. $y^+$ calculation).
\item \inltt{vorticity}: Computes the vorticity field.
\item \inltt{wss}: Computes wall shear stress field.
\end{enumerate}
The module list above can be seen by running the command
%
\begin{lstlisting}[style=BashInputStyle]
FieldConvert -l
\end{lstlisting}
%
In the following we will detail the usage of each module.
%
%
%

\subsection{Smooth the data: \textit{C0Projection} module}
To smooth the data of a given .fld file one can
use the \inltt{C0Projection} module of FieldConvert
%
\begin{lstlisting}[style=BashInputStyle]
FieldConvert -m C0Projection test.xml test.fld test-C0Proj.fld
\end{lstlisting}
%
where the file \inltt{test-C0Proj.fld} can be processed in a similar
way as described in section \ref{s:utilities:fieldconvert:sub:convert}
to visualise the result either in Tecplot, Paraview or VisIt.

The option \inltt{localtoglobalmap} will do a global gather of the
coefficients and then scatter them back to the local elements. This
will replace the coefficients shared between two elements with the
coefficients of one of the elements (most likely the one with the
highest id). Although not a formal projection it does not require any
matrix inverse and so is very cheap to perform.

The option \inltt{usexmlbcs} will enforce the boundary conditions
specified in the input xml file.

The option \inltt{helmsmoothing=$L$} will perform a Helmholtz
smoothing projection of the form
\[
\left (\nabla^2 + \left (\frac{2 \pi}{L}\right )^2 \right ) \hat{u}^{new} =
\left (\frac{2 \pi}{L}\right )^2 \hat{u}^{orig}
\]
which can be interpreted in a Fourier sense as smoothing the original
coefficients using a low pass filter of the form
\[
\hat{u}_k^{new} = \frac{1}{(1 + k^2/K_0^2)} \hat{u}_k^{orig} \mbox{\,  where  \,}
K_0 = \frac{2 \pi}{L}
\]
and so $L$ is the length scale below which the coefficients values are
halved or more. Since this form of the Helmholtz operator is not
possitive definite, currently a direct solver is necessary and so this
smoother is mainly of use in two-dimensions.

\subsection{Calculate Q-Criterion: \textit{QCriterion} module}
To perform the Q-criterion calculation and obtain an output
data containing the Q-criterion solution, the user can run
%
\begin{lstlisting}[style=BashInputStyle]
FieldConvert -m QCriterion test.xml test.fld test-QCrit.fld
\end{lstlisting}
%
where the file \inltt{test-QCrit.fld} can be processed in a similar
way as described in section \ref{s:utilities:fieldconvert:sub:convert}
to visualise the result either in Tecplot, Paraview or VisIt.
%
%
%

\subsection{Add composite ID: \textit{addcompositeid} module}
When dealing with a geometry that has many surfaces, we need to identify the
composites to assign boundary conditions. To assist in this, FieldConvert has a
\inltt{addcompositeid} module, which adds the composite ID of every element as a
new field. To use this we simply run
%
\begin{lstlisting}[style=BashInputStyle]
  FieldConvert -m addcompositeid mesh.xml out.dat
\end{lstlisting}
%
In this case, we have produced a Tecplot file which contains the mesh and a
variable that contains the composite ID. To assist in boundary identification,
the input file \inlsh{mesh.xml} should be a surface XML file that can be
obtained through the \mc \inltt{extract} module (see section
\ref{s:utilities:nekmesh:extract}).

\subsection{Sum two .fld files: \textit{addFld} module}
To sum two .fld files one can use the \inltt{addFld} module of FieldConvert
%
\begin{lstlisting}[style=BashInputStyle]
  FieldConvert -m addfld:fromfld=file1.fld:scale=-1 file1.xml file2.fld \
  file3.fld
\end{lstlisting}
%
In this case we use it in conjunction with the command \inltt{scale}
which multiply the values of a given .fld file by a constant \inltt{value}.
\inltt{file1.fld} is the file multiplied by \inltt{value}, \inltt{file1.xml}
is the associated session file, \inltt{file2.fld} is the .fld file which
is summed to \inltt{file1.fld} and finally \inltt{file3.fld} is the output
which contain the sum of the two .fld files.
\inltt{file3.fld} can be processed in a similar way as described 
in section \ref{s:utilities:fieldconvert:sub:convert} to visualise 
the result either in Tecplot, Paraview or VisIt.
%
%
%
\subsection{Combine two .fld files containing time averages: \textit{combineAvg} module}
To combine two .fld files obtained through the AverageFields or ReynoldsStresses filters,
use the \inltt{combineAvg} module of FieldConvert
%
\begin{lstlisting}[style=BashInputStyle]
  FieldConvert -m combineAvg:fromfld=file1.fld file1.xml file2.fld \
  file3.fld
\end{lstlisting}
%
\inltt{file3.fld} can be processed in a similar way as described 
in section \ref{s:utilities:fieldconvert:sub:convert} to visualise 
the result either in Tecplot, Paraview or VisIt.
%
%
%
\subsection{Concatenate two files: \textit{concatenate} module}
To concatenate \inltt{file1.fld} and \inltt{file2.fld} into \inltt{file-conc.fld}
one can run the following command
%
\begin{lstlisting}[style=BashInputStyle]
FieldConvert -m concatenate file.xml  file1.fld  file2.fld  file-conc.fld
\end{lstlisting}
%
where the file \inltt{file-conc.fld} can be processed in a similar
way as described in section \ref{s:utilities:fieldconvert:sub:convert}
to visualise the result either in Tecplot, Paraview or VisIt.
%
%
%
\subsection{Equi-spaced output of data: \textit{equispacedoutput} module}
This module interpolates the output data to an truly equispaced set of
points (not equispaced along the collapsed coordinate
system). Therefore a tetrahedron is represented by a tetrahedral
number of poinst. This produces much smaller output files. The points
are then connected together by simplices (triangles and tetrahedrons).

\begin{lstlisting}[style=BashInputStyle]
FieldConvert -m equispacedoutput test.xml test.fld test.dat
\end{lstlisting}

or

\begin{lstlisting}[style=BashInputStyle]
FieldConvert -m equispacedouttput test.xml test.fld test.vtu
\end{lstlisting}


\begin{notebox}
Currently this option is only set up for triangles, quadrilaterals,
tetrahedrons and prisms.
\end{notebox}

\subsection{Extract a boundary region: \textit{extract} module}
The boundary region of a domain can be extracted from the output
data using the following command line
%
\begin{lstlisting}[style=BashInputStyle]
FieldConvert -m extract:bnd=2:fldtoboundary=1 test.xml \
	test.fld test-boundary.fld
\end{lstlisting}
%
The option \inltt{bnd} specifies which boundary region to extract.
Note this is different to NekMesh where the parameter \inltt{surf}
is specified and corresponds to composites rather boundaries. If \inltt{bnd}
is not provided, all boundaries are extracted to different fields. The \inltt{fldtoboundary}
is an optional command argument which copies the expansion of test.fld into
the boundary region before outputting the .fld file. This option is on by default.
If it turned off using \inltt{fldtoboundary=0} the extraction will only evaluate the
boundary condition from the xml file. The output will be placed in test-boundary-b2.fld.
If more than one boundary region is specified the extension -b0.fld, -b1.fld etc will be
outputted. To process this file you will need an xml file of the same region.
This can be generated using the command:
%
\begin{lstlisting}[style=BashInputStyle]
NekMesh -m extract:surf=5  test.xml test-b0.xml
\end{lstlisting}
%
The surface to be extracted in this command is the composite
number and so needs to correspond to the boundary region
of interest. Finally to process the surface file one can use
%
\begin{lstlisting}[style=BashInputStyle]
FieldConvert  test-b0.xml test-b0.fld test-b0.dat
\end{lstlisting}
%
This will obviously generate a Tecplot output if a .dat file 
is specified as last argument. A .vtu extension will produce 
a Paraview or VisIt output.
%
%
%
\subsection{Compute the gradient of a field: \textit{gradient} module}
To compute the spatial gradients of all fields one can run the following command
%
\begin{lstlisting}[style=BashInputStyle]
FieldConvert -m gradient test.xml  test.fld  test-grad.fld
\end{lstlisting}
%
where the file \inltt{file-grad.fld} can be processed in a similar
way as described in section \ref{s:utilities:fieldconvert:sub:convert}
to visualise the result either in Tecplot, Paraview or VisIt.
%
%

\subsection{Extract a plane from 3DH1D expansion: \textit{homplane} module}

To obtain a 2D expansion containing one of the planes of a
3DH1D field file, use the command:
\begin{lstlisting}[style=BashInputStyle] 
FieldConvert -m homplane:planeid=value file.xml file.fld file-plane.fld
\end{lstlisting}

If the option \inltt{wavespace} is used, the Fourier coefficients
corresponding to \inltt{planeid} are obtained. The command in this case is:
\begin{lstlisting}[style=BashInputStyle] 
FieldConvert -m homplane:wavespace:planeid=value file.xml \
    file.fld file-plane.fld
\end{lstlisting}

The output file \inltt{file-plane.fld} can be processed in a similar 
way as described in section \ref{s:utilities:fieldconvert:sub:convert}
to visualise it either in Tecplot or in Paraview.

\subsection{Stretch a 3DH1D expansion: \textit{homstretch} module}

To stretch a 3DH1D expansion in the z-direction, use the command:
\begin{lstlisting}[style=BashInputStyle] 
FieldConvert -m homstretch:factor=value file.xml file.fld file-stretch.fld
\end{lstlisting}
The number of modes in the resulting field can be chosen using the command-line
parameter \inltt{output-points-hom-z}. Note that the output for
this module should always be a \inltt{.fld} file and this should not
be used in combination with other modules using a single command.

The output file \inltt{file-stretch.fld} can be processed in a similar 
way as described in section \ref{s:utilities:fieldconvert:sub:convert}
to visualise it either in Tecplot or in Paraview.


\subsection{Inner Product of a single or series of fields with respect to a single or series of fields: \textit{innerproduct} module}
You can take the inner product of one field with another field using
the following command:
\begin{lstlisting}[style=BashInputStyle]
  FieldConvert -m innerproduct:fromfld=file1.fld  file2.xml file2.fld \
  out.stdout
\end{lstlisting}
This command will load the \inltt{file1.fld} and \inltt{file2.fld}
assuming they both are spatially defined by \inltt{files.xml} and
determine the inner product of these fields. The input option
\inltt{fromfld} must therefore be specified in this module.

Optional arguments for this module are \inltt{fields} which allow you to specify
the fields that you wish to use for the inner product, i.e. 
\begin{lstlisting}[style=BashInputStyle]
  FieldConvert -m innerproduct:fromfld=file1.fld:fields=''0,1,2'' file2.xml \
  file2.fld out.stdout
\end{lstlisting}
will only take the inner product between the variables 0,1 and 2 in
the two fields files. The default is to take the inner product between
all fields provided.

Additional options include \inltt{multifldids} and \inltt{allfromflds}
which allow for a series of fields to be evaluated in the following
manner:
\begin{lstlisting}[style=BashInputStyle]
  FieldConvert -m innerproduct:fromfld=file1.fld:multifldids=''0-3''\
  file2.xml  file2.fld out.stdout
\end{lstlisting}
will take the inner product between a file names
field1\_0.fld, field1\_1.fld, field1\_2.fld and field1\_3.fld with
respect to field2.fld.

Analogously including the options \inltt{allfromflds}, i.e. 
\begin{lstlisting}[style=BashInputStyle]
  FieldConvert -m innerproduct:fromfld=file1.fld:multifldids=''0-3'':\
  allfromflds  file2.xml file2.fld out.stdout
\end{lstlisting}
Will take the inner product of all the from fields,
i.e. field1\_0.fld,field1\_1.fld,field1\_2.fld and field1\_3.fld with
respect to each other. This option essentially ignores file2.fld. Only
the unique inner products are evaluated so if four from fields are
given only the related trianuglar number $4\times5/2=10$ of inner
products are evaluated.

This option can be run in parallel. 

%
%
%

\subsection{Interpolate one field to another: \textit{interpfield} module}
To interpolate one field to another, one can use the following command:
%
\begin{lstlisting}[style=BashInputStyle]
FieldConvert -m interpfield:fromxml=file1.xml:fromfld=file1.fld \
	file2.xml file2.fld
\end{lstlisting}
%
This command will interpolate the field defined by \inltt{file1.xml}
and \inltt{file1.fld} to the new mesh defined in \inltt{file2.xml} and
output it to \inltt{file2.fld}.
The \inltt{fromxml} and \inltt{fromfld} must be specified in this module.
In addition there are two optional arguments \inltt{clamptolowervalue}
and \inltt{clamptouppervalue} which clamp the interpolation between
these two values. Their default values are -10,000,000 and 10,000,000.
%
\begin{tipbox}
This module can run in parallel where the speed is increased not
only due to using more cores but also, since the mesh is split into
smaller sub-domains, the search method currently adopted performs
faster.
\end{tipbox}
%
%
%
\subsection{Interpolate scattered point data to a field: \textit{interppointdatatofld} module}
\label{s:utilities:fieldconvert:sub:interppointdatatofld}
To interpolate discrete point data to a field, use the interppointdatatofld module:
%
\begin{lstlisting}[style=BashInputStyle]
FieldConvert -m interppointdatatofld file1.xml file1.pts file1.fld
\end{lstlisting}
%
This command will interpolate the data from \inltt{file1.pts} to the mesh
and expansions defined in \inltt{file1.xml} and output the field to \inltt{file1.fld}.
The file \inltt{file.pts} is of the form:
%
\begin{lstlisting}[style=XMLStyle]
<?xml version="1.0" encoding="utf-8" ?>
<NEKTAR>
  <POINTS DIM="1" FIELDS="a,b,c">
    1.0000 -1.0000 1.0000 -0.7778
    2.0000 -0.9798 0.9798 -0.7980
    3.0000 -0.9596 0.9596 -0.8182
    4.0000 -0.9394 0.9394 -0.8384
  </POINTS>
</NEKTAR>
\end{lstlisting}
%
where \inltt{DIM="1" FIELDS="a,b,c} specifies that the field is one-dimensional
and contains three variables, $a$, $b$, and $c$.
Each line defines a point, while the  first column contains its $x$-coordinate,
the second one contains the $a$-values, the third the $b$-values and so on.
In case of $n$-dimensional data, the $n$ coordinates are specified in the first $n$
columns accordingly.
%
In order to interpolate 1D data to a $n$D field, specify the matching coordinate in
the output field using the \inltt{interpcoord} argument:
%
\begin{lstlisting}[style=BashInputStyle]
FieldConvert -m interppointdatatofld:interppointdatatofld=1 3D-file1.xml \
		1D-file1.pts 3D-file1.fld
\end{lstlisting}
%
This will interpolate the 1D scattered point data from \inltt{1D-file1.pts} to the
$y$-coordinate of the 3D mesh defined in \inltt{3D-file1.xml}. The resulting field
will have constant values along the $x$ and $z$ coordinates.
For 1D Interpolation, the module implements a quadratic scheme and automatically
falls back to a linear method if only two data points are given.
A modified inverse distance method is used for 2D and 3D interpolation.
Linear and quadratic interpolation require the data points in the \inlsh{.pts}-file to be
sorted by their location in ascending order.
The Inverse Distance implementation has no such requirement.
%
%
%
\subsection{Interpolate a field to a series of points: \textit{interppoints} module}
You can interpolate one field to a series of given points using the following command:
\begin{lstlisting}[style=BashInputStyle]
FieldConvert -m interppoints:fromxml=file1.xml:fromfld=file1.fld \
        file2.pts file2.dat
\end{lstlisting}
This command will interpolate the field defined by \inltt{file1.xml} and
\inltt{file1.fld} to the points defined in \inltt{file2.pts} and output it to
\inltt{file2.dat}.
The \inltt{fromxml} and \inltt{fromfld} must be specified in this module.
The format of the file \inltt{file2.pts} is of the same form as for the
\textit{interppointdatatofld} module:
\begin{lstlisting}[style=XMLStyle]
<?xml version="1.0" encoding="utf-8" ?>
<NEKTAR>
  <POINTS DIM="2" FIELDS="">
    0.0 0.0
    0.5 0.0
    1.0 0.0
  </POINTS>
</NEKTAR>
\end{lstlisting}
There are three optional arguments \inltt{clamptolowervalue},
\inltt{clamptouppervalue} and \inltt{defaultvalue} the first two clamp the
interpolation between these two values and the third defines the default
value to be used if the point is outside the domain. Their default values
are -10,000,000, 10,000,000 and 0.

In addition, instead of specifying the file \inltt{file2.pts}, a module list of the form
\begin{lstlisting}[style=BashInputStyle]
FieldConvert -m interppoints:fromxml=file1.xml:fromfld= \
	file1.fld:line=npts,x0,y0,x1,y1
\end{lstlisting}
can be specified where \inltt{npts} is the number of equispaced points between
$(x0,y0)$ to $(x1,y1)$ which can also be used in 3D by specifying $(x0,y0,z0)$
to $(x1,y1,z1)$.

An extraction of a plane of points can also be specified by
\begin{lstlisting}[style=BashInputStyle] 
  FieldConvert -m interppoints:fromxml=file1.xml:fromfld=file1.fld:\
        plane=npts1,npts2,x0,y0,z0,x1,y1,z1,x2,y2,z2,x3,y3,z3
\end{lstlisting}
where \inltt{npts1,npts2} is the number of equispaced points in each
direction and $(x0,y0,z0)$, $(x1,y1,z1)$, $(x2,y2,z2)$ and $(x3,y3,z3)$
define the plane of points specified in a clockwise or anticlockwise direction.

In addition an extraction of a box of points can also be specified by
\begin{lstlisting}[style=BashInputStyle] 
  FieldConvert -m interppoints:fromxml=file1.xml:fromfld=file1.fld:\
       box=npts1,npts2,npts3,xmin,xmax,ymin,ymax,zmin,zmax
\end{lstlisting}
where \inltt{npts1,npts2,npts3} is the number of equispaced points in each 
direction and $(xmin,ymin,zmin)$ and $(xmax,ymax,zmax3)$ 
define the limits of the box of points. 

For the plane and box interpolation there is an additional optional
argument \inltt{cp=p0,q} which adds to the interpolated fields the value of
$c_p=(p-p0)/q$ and $c_{p0}=(p-p0+0.5 u^2)/q$ where $p0$ is a reference
pressure and $q$ is the free stream dynamics pressure. If the input
does not contain a field ``p'' or a velocity field ``u,v,w'' then $cp$
and $cp0$ are not evaluated accordingly
%
\begin{notebox} 
This module  runs in parallel for the plane and box extraction of points. In this case a series of .dat files are generated that can be concatinated together. Other options do not run in parallel.
\end{notebox}
%
%
%
\subsection{Isocontour extraction: \textit{iscontour} module}

Extract an isocontour from a field file. This option automatically
take the field to an equispaced distribution of points connected by
linear simplicies of triangles or tetrahedrons. The linear simplices
are then inspected to extract the isocontour of interest. To specify
the field \inltt{fieldid} can be provided giving the id of the field
of interest and \inltt{fieldvalue} provides the value of the
isocontour to be extracted.

\begin{lstlisting}[style=BashInputStyle]
  FieldConvert -m isocontour:fieldid=2:fieldvalue=0.5 test.xml test.fld \
          test-isocontour.dat
\end{lstlisting}

Alternatively \inltt{fieldstr}=''u+v'' can be specified to calculate
the field $u^2$ and extract its isocontour. You can also specify
\inltt{fieldname}=''UplusV'' to define the name of the isocontour in
the .dat file, i.e.
\begin{lstlisting}[style=BashInputStyle]
  FieldConvert -m isocontour:fieldstr="u+v":fieldvalue=0.5:\
    fieldname="UplusV" test.xml  test.fld test-isocontour.dat
\end{lstlisting}

Optionally \inltt{smooth} can be specified to smooth the isocontour
with default values \inltt{smoothnegdiffusion}=0.495,
\inltt{smoothnegdiffusion}=0.5 and \inltt{smoothiter}=100. This option
typically should be used wiht the \inltt{globalcondense} option which
removes multiply defined verties from the simplex definition which
arise as isocontour are generated element by element. The
\inltt{smooth} option preivously automatically called the
\inltt{globalcondense} option but this has been depracated since it is
now possible to read isocontour files directly and so it is useful to
have these as separate options.

In addition to the \inltt{smooth} or \inltt{globalcondense} options
you can specify \inltt{removesmallcontour}=100 which will remove
separate isocontours of less than 100 triangles. 

\begin{notebox}
Currently this option is only set up for triangles, quadrilaterals,
 tetrahedrons and prisms.
\end{notebox}
%
%
%
%

\subsection{Show high frequency energy of the Jacobian: \textit{jacobianenergy} module}

\begin{lstlisting}[style=BashInputStyle]
FieldConvert -m jacobianenergy file.xml file.fld jacenergy.fld
\end{lstlisting}

The option \inltt{topmodes} can be used to specify the number of top modes to
keep.

The output file \inltt{jacenergy.fld} can be processed in a similar
way as described in section \ref{s:utilities:fieldconvert:sub:convert}
to visualise the result either in Tecplot, Paraview or VisIt. 

\subsection{Calculate mesh quality: \textit{qualitymetric} module}

The \inltt{qualitymetric} module assesses the quality of the mesh by calculating
a per-element quality metric and adding an additional field to any resulting
output. This does not require any field input, therefore an example usage looks
like

\begin{lstlisting}[style=BashInputStyle]
FieldConvert -m qualitymetric mesh.xml mesh-with-quality.dat
\end{lstlisting}

Two quality metrics are implemented that produce scalar fields $Q$:

\begin{itemize}
  \item By default a metric outlined in~\cite{GaRoPeSa15} is produced, where all
  straight sided elements have quality $Q = 1$ and $Q < 1$ shows the deformation
  between the curved element and the straight-sided element. If $Q = 0$ then the
  element is invalid. Note that $Q$ varies over the volume of the element but is
  not guaranteed to be continuous between elements.
  \item Alternatively, if the \inlsh{scaled} option is passed through to the
  module, then the scaled Jacobian
  \[
    J_s =
    \frac{\min_{\xi\in\Omega_{\text{st}}}J(\xi)}{\max_{\xi\in\Omega_{\text{st}}}J(\xi)}
  \]
  (i.e. the ratio of the minimum to maximum Jacobian of each element) is
  calculated. Again $Q = 1$ denotes an ideal element, but now invalid elements
  are shown by $Q < 0$. Any elements with $Q$ near zero are determined to be low
  quality.
\end{itemize}


%
%
%

\subsection{Extract mean mode of 3DH1D expansion: \textit{meanmode} module}

To obtain a 2D expansion containing the mean mode (plane zero in Fourier space) of a
3DH1D field file, use the command:
\begin{lstlisting}[style=BashInputStyle] 
FieldConvert -m meanmode file.xml file.fld file-mean.fld
\end{lstlisting}

The output file \inltt{file-mean.fld} can be processed in a similar 
way as described in section \ref{s:utilities:fieldconvert:sub:convert}
to visualise the result either in Tecplot or in Paraview or VisIt.
%
%
%

\subsection{ Project point data to a field: \textit{pointdatatofld} module}
\label{s:utilities:fieldconvert:sub:pointdatatofld}
To project a series of points given at the same quadrature distribution as the .xml file and write out a .fld file use the pointdatatofld module:
%
\begin{lstlisting}[style=BashInputStyle]
FieldConvert --noequispaced -m pointdatatofld file.pts file.xml file.fld
\end{lstlisting}
%
This command will read in the points provided in the \inltt{file.pts}
and assume these are given at the same quadrature distribution as the
mesh and expansions defined in \inltt{file.xml} and output the field
to \inltt{file.fld}. If the points do not match an error will be dumped. 


The file \inltt{file.pts} which is assumed to be given by an interpolation from another source is of the form:
%
\begin{lstlisting}[style=XMLStyle]
<?xml version="1.0" encoding="utf-8" ?>
<NEKTAR>
  <POINTS DIM="3" FIELDS="p">
  1.70415 -0.4       -0.0182028      -0.106893
  1.70415 -0.395683  -0.0182028      -0.106794
  1.70415 -0.3875    -0.0182028      -0.106698
  1.70415 -0.379317  -0.0182028      -0.103815
  </POINTS>
</NEKTAR>
\end{lstlisting}
%
where \inltt{DIM="3" FIELDS="p} specifies that the field is
three-dimensional and contains one variable, $p$. Each line defines a
point, the first, second, and third columns contains the
$x,y,z$-coordinate and subsequent columns contain the field values, in
this case the $p$-value So in the general case of $n$-dimensional
data, the $n$ coordinates are specified in the first $n$ columns
accordingly followed by the field data. 

The default argument is to use the equipapced (but potentially
collapsed) coordinates which can  be obtained from the command.

\begin{lstlisting}[style=BashInputStyle]
FieldConvert file.xml file.dat
\end{lstlisting}

In this case the pointdatatofld module shoudl be used without the
\inltt{--noequispaced} option. However this can lead to problems when
peforming an elemental forward projection/transform since the mass
matrix in a deformed element can be singular as the equispaced points
do not have a sufficiently accurate quadrature rule that spans the
polynomial space. Therefore it is adviseable to use the set of points
given by

\begin{lstlisting}[style=BashInputStyle]
FieldConvert --noequispaced file.xml file.dat
\end{lstlisting}

which produces a set of points at the gaussian collapsed
coordinates. In this case one must also use the \inltt{--noequispaced}
option when projecting to a field.

Finally the option \inltt{setnantovalue=0} can also be used which sets
any nan values in the interpolation to zero or any specified value in
this option.

%
%
%

\subsection{Print L2 and LInf norms: \textit{printfldnorms} module}

\begin{lstlisting}[style=BashInputStyle] 
FieldConvert -m printfldnorms test.xml test.fld out.stdout
\end{lstlisting}

This module does not create an output file which is reinforced by the
out.stdout option. The L2 and LInf norms for each field variable are
then printed to the stdout.
%
%
%

\subsection{Computes the scalar gradient: \textit{scalargrad} module}
The scalar gradient of a field is computed by running:
\begin{lstlisting}[style=BashInputStyle]
FieldConvert -m scalargrad:bnd=0 test.xml test.fld test-scalgrad.fld
\end{lstlisting}
The option \inltt{bnd} specifies which boundary region to extract. Note this is different to NekMesh where the parameter \inltt{surf} is specified and corresponds to composites rather boundaries. If \inltt{bnd} is not provided, all boundaries are extracted to different fields. To process this file you will need an xml file of the same region.

%
%
%

\subsection{Scale a given .fld: \textit{scaleinputfld} module}
To scale a .fld file by a given scalar quantity, the user can run:
\begin{lstlisting}[style=BashInputStyle]
FieldConvert -m scaleinputfld:scale=value test.xml test.fld test-scal.fld
\end{lstlisting}
The argument \inltt{scale=value} rescales of a factor \inltt{value}
\inltt{test.fld} by the factor value.
The output file \inltt{file-conc.fld} can be processed in a similar
way as described in section \ref{s:utilities:fieldconvert:sub:convert}
to visualise the result  either in Tecplot, Paraview or VisIt.

%
%
%
\subsection{Time-averaged shear stress metrics: \textit{shear} module}
Time-dependent wall shear stress derived metrics relevant to cardiovascular fluid dynamics research can be computed using this module. They are

\begin{itemize}
\item TAWSS: time-averaged wall shear stress;
\item OSI: oscillatory shear index;
\item transWSS: transverse wall shear stress;
\item TACFI: time-averaged cross-flow index;
\item TAAFI: time-averaged aneurysm formation index;
\item |WSSG|: wall shear stress gradient.
\end{itemize}

To compute these, the user can run:
\begin{lstlisting}[style=BashInputStyle]
FieldConvert -m shear:N=value:fromfld=test_id_b0.fld test.xml test-multishear.fld
\end{lstlisting}
The argument \inltt{N} and \inltt{fromfld} are compulsory arguments that respectively define the number of \inltt{fld} files corresponding to the number of discrete equispaced time-steps, and the first \inltt{fld} file which should have the form of \inltt{test\_id\_b0.fld} where the first underscore in the name marks the starting time-step file ID.

The input \inltt{.fld} files are the outputs of the \textit{wss} module. If they do not contain the surface normals (an optional output of the \textit{wss} modle), then the \textit{shear} module will not compute the last metric, |WSSG|.


%
%
%
\subsection{Boundary layer height calculation: \textit{surfdistance} module}

The surface distance module computes the height of a prismatic boundary layer
and projects this value onto the surface of the boundary, in a similar fashion
to the \inltt{extract} module. In conjunction with a mesh of the surface, which
can be obtained with \inltt{NekMesh}, and a value of the average wall shear
stress, one potential application of this module is to determine the
distribution of $y^+$ grid spacings for turbulence calculations.

To compute the height of the prismatic layer connected to boundary region 3, the
user can issue the command:
\begin{lstlisting}[style=BashInputStyle]
FieldConvert -m surfdistance:bnd=3 input.xml output.fld
\end{lstlisting}
Note that no \inltt{.fld} file is required, since the mesh is the only input
required in order to calculate the element height. This produces a file
\inltt{output\_b3.fld}, which can be visualised with the appropriate surface
mesh from \inltt{NekMesh}.

%
%
%
\subsection{Calculate vorticity: \textit{vorticity} module}
To perform the vorticity calculation and obtain an output
data containing the vorticity solution, the user can run
\begin{lstlisting}[style=BashInputStyle]
FieldConvert -m vorticity test.xml test.fld test-vort.fld
\end{lstlisting}
where the file \inltt{test-vort.fld} can be processed in a similar
way as described in section \ref{s:utilities:fieldconvert:sub:convert}.
%
%
%

\subsection{Computing the wall shear stress: \textit{wss} module}
To obtain the wall shear stres vector and magnitude, the user can run:
\begin{lstlisting}[style=BashInputStyle]
FieldConvert -m wss:bnd=0:addnormals=1 test.xml test.fld test-wss.fld
\end{lstlisting}
The option \inltt{bnd} specifies which boundary region to extract. Note this is different to NekMesh where the parameter \inltt{surf} is specified and corresponds to composites rather boundaries. If \inltt{bnd} is not provided, all boundaries are extracted to different fields. The \inltt{addnormals} is an optional command argument which, when turned on, outputs the normal vector of the extracted boundary region as well as the shear stress vector and magnitude. This option is off by default. To process the output file(s) you will need an xml file of the same region.
%
%
%

\subsection{Manipulating meshes with FieldConvert}
FieldConvert has support for two modules that can be used in conjunction with
the linear elastic solver, as shown in chapter~\ref{s:elasticity}. To do this,
FieldConvert has an XML output module, in addition to the Tecplot and VTK
formats.

The \inltt{deform} module, which takes no options, takes a displacement field
and applies it to the geometry, producing a deformed mesh:
\begin{lstlisting}[style=BashInputStyle]
FieldConvert -m deform input.xml input.fld deformed.xml
\end{lstlisting}

The \inltt{displacement} module is designed to create a boundary condition field
file. Its intended use is for mesh generation purposes. It can be used to
calculate the displacement between the linear mesh and a high-order surface, and
then produce a \inltt{fld} file, prescribing the displacement at the boundary,
that can be used in the linear elasticity solver.

Presently the process is somewhat convoluted and must be used in conjunction
with NekMesh to create the surface file. However the bash input below
describes the procedure. Assume the high-order mesh is in a file called
\inlsh{mesh.xml}, the linear mesh is \inlsh{mesh-linear.xml} that can be
generated by removing the \inltt{CURVED} section from \inlsh{mesh.xml}, and that
we are interested in the surface with ID 123.

\begin{lstlisting}[style=BashInputStyle]
# Extract high order surface
NekMesh -m extract:surf=123 mesh.xml mesh-surf-curved.xml

# Use FieldConvert to calculate displacement between two surfaces
FieldConvert -m displacement:id=123:to=mesh-surf-curved.xml \
    mesh-linear.xml mesh-deformation.fld

# mesh-deformation.fld is used as a boundary condition inside the
# solver to prescribe the deformation conditions.xml contains
# appropriate Nektar++ parameters (mu, E, other BCs, ...)
LinearElasticSolver mesh-linear.xml conditions.xml

# This produces the final field mesh-linear.fld which is the
# displacement field, use FieldConvert to apply it:
FieldConvert-g -m deform mesh-linear.xml mesh-linear.fld mesh-deformed.xml
\end{lstlisting}

\section{FieldConvert in parallel}
To run FieldConvert in parallel the user needs to compile
\nekpp with MPI support and can employ the following
command
\begin{lstlisting}[style=BashInputStyle]
mpirun -np <nprocs> FieldConvert test.xml test.fld test.dat
\end{lstlisting}
or
\begin{lstlisting}[style=BashInputStyle]
mpirun -np <nprocs> FieldConvert test.xml test.fld test.vtu
\end{lstlisting}
replacing \inltt{<nprocs>} with the number of processors.  This will
produce multiple \inltt{.dat} or \inltt{.vtu} files of the form
\inltt{test\_P0.dat}, \inltt{test\_P1.dat}, \inltt{test\_P2.dat} or
\inltt{test\_P0.vtu}, \inltt{test\_P1.vtu}, \inltt{test\_P2.vtu}. Note
when producing a parallel \inltt{.vtu} file an additional file called
\inltt{.pvtu} is written out which allows for parallel reading of the
individual \inltt{.vtu} files. Similarly functions that produce a
\inltt{.fld} file output can be processed in this manner. In the case
when producing a .fld file a directory called \inltt{test.fld} (or the
specified output name) will be produced with the standard parallel
field files placed within the directory.
%
%
%
\section{Processing large files in serial}
When processing large files, it is not always convenient to run in parallel but
process each parallel partition in serial, for example when interpolating a
solution field from one mesh to another.

\subsection{Using the \texttt{nprocs} and \texttt{procid} options}

One option is to use the \inltt{--nprocs} and \inltt{--procid} command line
options. For example, the following command will interpolate partition 2 of a
decomposition into 10 partitions of \inltt{fiile2.xml} from \inltt{file1.fld}
\begin{lstlisting}[style=BashInputStyle]
FieldConvert --nprocs 10 --procid 2 \
        -m interpfield:fromxml=file1.xml:fromfld=file1.fld \
        file2.xml file2.fld
\end{lstlisting}
This call will only therefore consider the interpolation process across one
partition (namely, partition 2). To create the full interpolated field requires
a loop over each of the partitions, which, in a bash shell can be run as
\begin{lstlisting}[style=BashInputStyle] 
for n in `seq 0 9`; do
    FieldConvert --nprocs 10 --procid $n \
            -m interpfield:fromxml=file1.xml:fromfld=file1.fld \
            file2.xml file2.fld
done
\end{lstlisting}
The resulting output will lie in a directory called \inltt{file2.fld}, with each
of the different parallel partitions in files with names \inltt{P0000000.fld},
\inltt{P0000001.fld}, \dots, \inltt{P0000009.fld}. This is nearly a complete
parallel field file. However, the \inltt{Info.xml} file, which contains the
information about which elements lie in each partition, is missing. This can be
generated by using the command
\begin{lstlisting}[style=BashInputStyle] 
FieldConvert --nprocs 10 file2.xml file2.fld/Info.xml:info
\end{lstlisting}
Note the final \inltt{:info} extension on the last argument is necessary to tell
FieldConvert that you wish to generate an info file, but with the extension
\inltt{.xml}. This syntax allows the routine not to get confused with the
input/output XML files.

\subsection{Using the --part-only and --part-only-overlapping options}

Another approach to serially proessing a large file is to initially process the
file into multiple partitions. This can be done with the \inltt{--part-only}
option. So the command
\begin{lstlisting}[style=BashInputStyle] 
FieldConvert --part-only 10 file.xml file.fld
\end{lstlisting}
will partition the mesh into 10 paritions and write each partition into a
directory called \inltt{file\_xml}. If you enter this directory you will find
partitioned XML files \inltt{P0000000.xml}, \inltt{P0000001.xml}, \dots,
\inltt{P0000009.xml} which can then be processed individually as outlined above.

There is also a \inltt{--part-only-overlapping} option, which can be run in the
same fashion.
\begin{lstlisting}[style=BashInputStyle] 
FieldConvert --part-only-overlapping 10 file.xml file.fld
\end{lstlisting}
In this mode, the mesh is partitioned into 10 partitions in a similar manner,
but the elements at the partition edges will now overlap, so that the
intersection of each partition with its neighbours is non-empty. This is
sometime helpful when, for example, producing a global isocontour which has been
smoothed. Applying the smoothed isocontour extraction routine with the
\inltt{--part-only} option will produce a series of isocontour where there will
be a gap between partitions, as the smoother tends to shrink the isocontour
within a partition. using the \inltt{--part-only-overlapping} option will still
yield a shrinking isocontour, but the overlapping partitions help to overlap the
partiiton boundaries.

%%% Local Variables:
%%% mode: latex
%%% TeX-master: "../user-guide"
%%% End:
