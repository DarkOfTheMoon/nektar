\section{Installing from Source}
\label{s:installation:source}

This section explains how to build Nektar++ from the source-code package.

Nektar++ uses a number of third-party libraries. Some of these are required,
others are optional. It is generally more straightforward to use versions of
these libraries supplied pre-packaged for your operating system, but if you run
into difficulties with compilation errors or failing regression tests, the
Nektar++ build system can automatically build tried-and-tested versions of these
libraries for you. This requires enabling the relevant options in the CMake
configuration.


\subsection{Obtaining the source code}
There are two ways to obtain the source code for \nekpp:
\begin{itemize}
	\item Download the latest source-code archive from the
	\href{http://www.nektar.info/downloads}{Nektar++ downloads page}.
	\item Clone the git repository
	\begin{itemize}
	\item Using anonymous access. This does not require
	credentials but any changes to the code cannot be pushed to the
	public repository. Use this initially if you would like to try using
	Nektar++ or make local changes to the code.
    \begin{lstlisting}[style=BashInputStyle]
        git clone http://gitlab.nektar.info/clone/nektar/nektar.git nektar++
    \end{lstlisting}
	\item Using authenticated access. This will allow you to directly contribute
	back into the code.
    \begin{lstlisting}[style=BashInputStyle]
        git clone git@gitlab.nektar.info:nektar/nektar.git nektar++
    \end{lstlisting}
    \begin{tipbox}
    You can easily switch to using the authenticated access from anonymous
    access at a later date.
    \end{tipbox}
	\end{itemize}
\end{itemize}

\subsection{Linux}
\subsubsection{Prerequisites}
\nekpp uses a number of external programs and libraries for some or all of its
functionality. Some of these are \emph{required} and must be installed prior to
compiling Nektar++, most of which are available as pre-built \emph{system}
packages on most Linux distributions or can be installed manually by a
\emph{user}. Others are optional and required only for specific features, or can
be downloaded and compiled for use with Nektar++ \emph{automatically} (but not
installed system-wide).

\begin{center}
\begin{tabular}{lccccl}
\toprule
        &      & \multicolumn{3}{c}{Installation} & \\ \cmidrule(r){3-5}
Package & Req. & Sys. & User & Auto.              & Note \\
\midrule
C++ compiler    & \cmark & \cmark & & & gcc, icc, etc \\
CMake  $>2.8.7$ & \cmark & \cmark & \cmark &        & Ncurses
GUI optional
\\
BLAS            & \cmark & \cmark & \cmark &        & Or MKL,
ACML, OpenBLAS
\\
LAPACK          & \cmark & \cmark & \cmark &        & \\
Boost $>=1.52$   & \cmark & \cmark & \cmark & \cmark & Compile
with iostreams
\\
ModMETIS        & \cmark &        &        & \cmark & \\
FFTW $>3.0$     &        & \cmark & \cmark & \cmark & For
high-performance FFTs\\
ARPACK $>2.0$   &        & \cmark & \cmark &        & For
arnoldi algorithms\\
OpenMPI         &        & \cmark &        &        & For
parallel execution\\
GSMPI           &        &        &        & \cmark & For
parallel execution\\
HDF5            &        & \cmark & \cmark & \cmark & For
large-scale parallel I/O (requires CMake >3.1)\\
PETSc           &        &        & \cmark & \cmark &
Alternative linear solvers\\
Scotch          &        & \cmark & \cmark & \cmark &
Alternative mesh partitioning\\
VTK $>5.8$      &        & \cmark & \cmark &        & Visualisation
utilities\\
\bottomrule
\end{tabular}
\end{center}

\begin{warningbox}
Boost version 1.51 has a bug which prevents \nekpp working correctly.
Please use a newer version.
\end{warningbox}


\subsubsection{Quick Start}
Open a terminal.

If you have downloaded the tarball, first unpack it:
\begin{lstlisting}[style=BashInputStyle]
tar -zxvf nektar++-(*\nekver*).tar.gz
\end{lstlisting}
Change into the \inlsh{nektar++} source code directory
\begin{lstlisting}[style=BashInputStyle]
    mkdir -p build && cd build
    ccmake ../
    make install
\end{lstlisting}

\subsubsection{Detailed instructions}
From a terminal:
\begin{enumerate}
    \item If you have downloaded the tarball, first unpack it
    \begin{lstlisting}[style=BashInputStyle]
tar -zxvf nektar++-(*\nekver*).tar.gz
    \end{lstlisting}

    \item Change into the source-code directory, create a \inltt{build}
    subdirectory and enter it
    \begin{lstlisting}[style=BashInputStyle]
    mkdir -p build && cd build
    \end{lstlisting}

    \item Run the CMake GUI and configure the build by pressing \inltt{c}
    \begin{lstlisting}[style=BashInputStyle]
    ccmake ../
    \end{lstlisting}
    \begin{itemize}
        \item Select the components of Nektar++ (prefixed with
        \inltt{NEKTAR\_BUILD\_}) you would like to build. Disabling solvers
        which you do not require will speed up the build process.
        \item Select the optional libraries you would like to use (prefixed with
        \inltt{NEKTAR\_USE\_}) for additional functionality.
        \item Select the libraries not already available on your system which
        you wish to be compiled automatically (prefixed with
        \inltt{THIRDPARTY\_BUILD\_})
    \end{itemize}
    A full list of configuration options can be found in
    Section~\ref{s:installation:source:cmake}.

    \begin{notebox}
    Selecting \inltt{THIRDPARTY\_BUILD\_} options will request CMake to
    automatically download thirdparty libraries and compile them within the
    \nekpp directory. If you have administrative access to your machine, it is
    recommended to install the libraries system-wide through your
    package-management system.
    \end{notebox}


    \item Press \inltt{c} to configure the build. If errors arise relating to
    missing libraries, review the \inltt{THIRDPARTY\_BUILD\_} selections in
    the configuration step above or install the missing libraries manually or
    from system packages.

    \item When configuration completes without errors, press \inltt{c} again
    until the option \inltt{g} to generate build files appears. Press \inltt{g}
    to generate the build files and exit CMake.

    \item Compile the code
    \begin{lstlisting}[style=BashInputStyle]
        make install
    \end{lstlisting}
    During the build, missing third-party libraries will be automatically
    downloaded, configured and built in the \nekpp \inlsh{build} directory.

    % Hacky way to get an lstlisting to an argument of a macro
    \newsavebox\installationLinuxTip
    \begin{lrbox}{\installationLinuxTip}\begin{minipage}{0.8\linewidth}
    \begin{lstlisting}[style=BashInputStyle]
    make -j4 install
    \end{lstlisting}
    \end{minipage}
    \end{lrbox}

    \begin{tipbox}
    If you have multiple processors/cores on your system, compilation can be
    significantly increased by adding the \inlsh{-jX} option to make, where X is
    the number of simultaneous jobs to spawn. For example, use

    \noindent\usebox\installationLinuxTip

    on a quad-core system.
    \end{tipbox}

    \item Test the build by running unit and regression tests.
    \begin{lstlisting}[style=BashInputStyle]
    ctest
    \end{lstlisting}
\end{enumerate}

\subsection{OS X}

\subsubsection{Prerequisites}
\nekpp uses a number of external programs and libraries for some or all of its
functionality. Some of these are \emph{required} and must be installed prior to
compiling Nektar++, most of which are available on \emph{MacPorts}
(www.macports.org) or can be installed manually by a \emph{user}. Others are
optional and required only for specific features, or can be downloaded and
compiled for use with Nektar++ \emph{automatically} (but not installed
system-wide).

\begin{notebox}
  To compile \nekpp on OS X, Apple's Xcode Developer Tools must be
  installed. They can be installed either from the App Store (only on Mac OS
  10.7 and above) or downloaded directly from
  \href{http://connect.apple.com/}{http://connect.apple.com/} (you are required
  to have an Apple Developer Connection account).  Xcode includes Apple
  implementations of BLAS and LAPACK (called the Accelerate Framework).
\end{notebox}

\begin{center}
\begin{tabular}{lccccl}
\toprule
        &      & \multicolumn{3}{c}{Installation} & \\ \cmidrule(r){3-5}
Package & Req. & MacPorts & User & Auto.          & Note \\
\midrule
Xcode           & \cmark &        & & & Provides developer tools \\
CMake  $>2.8.7$ & \cmark & \texttt{cmake}   & \cmark &        & Ncurses
GUI optional \\
BLAS            & \cmark &                  &        &        & Part of
Xcode \\
LAPACK          & \cmark &                  &        &        & Part of
Xcode \\
Boost $>=1.52$   & \cmark & \texttt{boost}   & \cmark & \cmark & Compile
with iostreams \\
TinyXML         & \cmark & \texttt{tinyxml} & \cmark & \cmark & \\
ModMETIS        & \cmark &                  &        & \cmark & \\
FFTW $>3.0$     &        & \texttt{fftw-3}  & \cmark & \cmark & For
high-performance FFTs\\
ARPACK $>2.0$   &        & \texttt{arpack}  & \cmark &        & For
arnoldi algorithms\\
OpenMPI         &        & \texttt{openmpi} &        &        & For
parallel execution\\
GSMPI           &        &                  &        & \cmark & For
parallel execution\\
HDF5            &        & \cmark & \cmark & \cmark & For
large-scale parallel I/O (requires CMake >3.1)\\
PETSc           &        & \texttt{petsc}   & \cmark & \cmark &
Alternative linear solvers\\
Scotch          &        & \texttt{scotch}  & \cmark & \cmark &
Alternative mesh partitioning\\
VTK $>5.8$      &        & \texttt{vtk}     & \cmark &        &
Visualisation utilities\\
\bottomrule
\end{tabular}
\end{center}

\newsavebox\installationOSXMacPortsTip
\begin{lrbox}{\installationOSXMacPortsTip}\begin{minipage}{0.8\linewidth}
\begin{lstlisting}[style=BashInputStyle]
sudo port install cmake
\end{lstlisting}
\end{minipage}
\end{lrbox}

\begin{tipbox}
CMake, and some other software, is available from MacPorts
(\url{http://macports.org}) and can be installed using, for example,

\noindent\usebox\installationOSXMacPortsTip

Package names are given in the table above. Similar packages also exist in other
package managers such as Homebrew.
\end{tipbox}


\subsubsection{Quick Start}
Open a terminal (Applications->Utilities->Terminal).

If you have downloaded the tarball, first unpack it:
\begin{lstlisting}[style=BashInputStyle]
tar -zxvf nektar++-(*\nekver*).tar.gz
\end{lstlisting}
Change into the \inlsh{nektar++} source code directory
\begin{lstlisting}[style=BashInputStyle]
    mkdir -p build && cd build
    ccmake ../
    make install
\end{lstlisting}

\subsubsection{Detailed instructions}
From a terminal (Applications->Utilities->Terminal):
\begin{enumerate}
    \item If you have downloaded the tarball, first unpack it
    \begin{lstlisting}[style=BashInputStyle]
tar -zxvf nektar++-(*\nekver*).tar.gz
    \end{lstlisting}

    \item Change into the source-code directory, create a \inltt{build}
    subdirectory and enter it
    \begin{lstlisting}[style=BashInputStyle]
    mkdir -p build && cd build
    \end{lstlisting}

    \item Run the CMake GUI and configure the build
    \begin{lstlisting}[style=BashInputStyle]
    ccmake ../
    \end{lstlisting}
    Use the arrow keys to navigate the options and \inlsh{ENTER} to select/edit
    an option.
    \begin{itemize}
        \item Select the components of Nektar++ (prefixed with
        \inltt{NEKTAR\_BUILD\_}) you would like to build. Disabling solvers
        which you do not require will speed up the build process.
        \item Select the optional libraries you would like to use (prefixed with
        \inltt{NEKTAR\_USE\_}) for additional functionality.
        \item Select the libraries not already available on your system which
        you wish to be compiled automatically (prefixed with
        \inltt{THIRDPARTY\_BUILD\_})
        \item
    \end{itemize}
    A full list of configuration options can be found in
    Section~\ref{s:installation:source:cmake}.

    \begin{notebox}
    Selecting \inltt{THIRDPARTY\_BUILD\_} options will request CMake to
    automatically download thirdparty libraries and compile them within the
    \nekpp directory. If you have administrative access to your machine, it is
    recommended to install the libraries system-wide through MacPorts.
    \end{notebox}

    \item Press \inltt{c} to configure the build. If errors arise relating to
        missing libraries (variables set to \inlsh{NOTFOUND}), review the
        \inltt{THIRDPARTY\_BUILD\_} selections in the previous
    step or install the missing libraries manually or through MacPorts.

    \item When configuration completes without errors, press \inltt{c} again
    until the option \inltt{g} to generate build files appears. Press \inltt{g}
    to generate the build files and exit CMake.

    \item Compile the code
    \begin{lstlisting}[style=BashInputStyle]
        make install
    \end{lstlisting}
    During the build, missing third-party libraries will be automatically
    downloaded, configured and built in the \nekpp \inlsh{build} directory.

    % Hacky way to get an lstlisting to an argument of a macro
    \newsavebox\installationMacTip
    \begin{lrbox}{\installationMacTip}\begin{minipage}{0.8\linewidth}
    \begin{lstlisting}[style=BashInputStyle]
    make -j4 install
    \end{lstlisting}
    \end{minipage}
    \end{lrbox}

    \begin{tipbox}
    If you have multiple processors/cores on your system, compilation can be
    significantly increased by adding the \inlsh{-jX} option to make, where X is
    the number of simultaneous jobs to spawn. For example,
    \noindent\usebox\installationMacTip
    \end{tipbox}

    \item Test the build by running unit and regression tests.
    \begin{lstlisting}[style=BashInputStyle]
    ctest
    \end{lstlisting}
\end{enumerate}

\subsection{Windows}

Windows compilation is supported, but the build process is somewhat convoluted
at present. As such, only serial execution is supported with a minimal amount of
additional build packages. These can either be installed by the user, or
automatically in the build process.

\begin{center}
\begin{tabular}{lcccl}
\toprule
                  &        & \multicolumn{2}{c}{Installation} & \\ \cmidrule(r){3-4}
Package           & Req.   & User   & Auto.        & Note \\
\midrule
MS Visual Studio  & \cmark & \cmark &              & 2012, 2013 and 2015 known working\\
CMake $\geq 3.0$  & \cmark & \cmark &              & \\
BLAS              & \cmark & \cmark & \cmark       & \\
LAPACK            & \cmark & \cmark & \cmark       & \\
Boost $\geq 1.55$ & \cmark & \cmark & \cmark       & Compile with iostreams\\
ModMETIS          & \cmark & \cmark & \cmark       & \\
\bottomrule
\end{tabular}
\end{center}

\subsubsection{Detailed instructions}

\begin{enumerate}
  \item Install Microsoft Visual Studio 2015 (preferred), 2013 or 2012 (known to
  work). This can be obtained from Microsoft free of charge by using their
  Community developer tools from
  \url{https://www.visualstudio.com/en-us/products/visual-studio-community-vs.aspx}.
  \item Install WinRAR from \url{http://www.rarlab.com/download.htm}.
  \item Install CMake 3.0+ from \url{http://www.cmake.org/download/}. When
  prompted, select the option to add CMake to the system PATH.
  \item (Optional) Install Git from \url{http://git-scm.com/download/win} to use
  the development versions of \nekpp. When prompted, select the option to add
  Git to the system PATH. You do not need to select the option to add Unix tools
  to the PATH.
  \item (Optional) If you do not wish to build boost during the compilation
  process (which can take some time), then boost binaries can be found at
  \url{http://sourceforge.net/projects/boost/files/boost-binaries/1.61.0/}. By
  default these install into \inlsh{C:\textbackslash local\textbackslash
    boost\_1\_61\_0}. If you use these libraries, you will need to:
  \begin{itemize}
    \item Rename \texttt{libs-msvc14.0} to \texttt{lib}
    \item Inside the \texttt{lib} directory, create duplicates of
    \texttt{boost\_zlib.dll} and \texttt{boost\_bzip2.dll} called
    \texttt{zlib.dll} and \texttt{libbz2.dll}
	\item Add a \inlsh{BOOST\_HOME} environment variable. To do so,
	navigate to \emph{Control Panel > System and Security > System}, select
	\emph{Advanced System Settings}, and in the \emph{Advanced} tab click the
	\emph{Environment Variables}. In the \emph{System Variables} box, click
	\emph{New}. In the \emph{New System Variable} window, type \inlsh{BOOST\_HOME}
	next to \emph{Variable name} and \inlsh{C:\textbackslash local\textbackslash
	boost\_1\_61\_0} next to\emph{Variable value}.
  \end{itemize}
  \item Unpack \inlsh{nektar++-\nekver.tar.gz} using WinRAR.

  \begin{notebox}
    Some Windows versions do not recognise the path of a folder which has
    \inltt{++} in the name. If you think that your Windows version can not
    handle path containing special characters, you should rename
    \inlsh{nektar++-\nekver} to \inlsh{nektar-\nekver}.
  \end{notebox}
  \item Create a \inlsh{builds} directory within the \inlsh{nektar++-\nekver}
  subdirectory.
  \item Open a Visual Studio terminal. From the start menu, this can be found in
  \emph{All Programs > Visual Studio 2015 > Visual Studio Tools > Developer
    Command Prompt for VS2015}.
  \item Change directory into the \texttt{builds} directory and run the CMake
  graphical utility:
  \begin{lstlisting}[style=BashInputStyle]
    cd C:\path\to\nektar\builds
    cmake-gui ..
  \end{lstlisting}
  \item Select the build system you want to generate build scripts for. Note
  that \emph{Visual Studio 2015} is listed as \emph{Visual Studio 14} in the
  drop-down list. If you have a 64-bit installation of Windows, you should
  select the \emph{Win64} variant, otherwise 32-bit executables will be
  generated. Select the option to use the native compilers.
  \item Click the \emph{Configure} button, then the \emph{Generate} button.
  \item Return to the command line and issue the command:
  \begin{lstlisting}[style=BashInputStyle]
    msbuild INSTALL.vcxproj /p:Configuration=Release
  \end{lstlisting}
  To build in parallel with, for example, 12 processors, issue:
  \begin{lstlisting}[style=BashInputStyle]
    msbuild INSTALL.vcxproj /p:Configuration=Release /m:12
  \end{lstlisting}
  \item After the installation process is completed, the executables will be
  available in \inlsh{builds\textbackslash dist\textbackslash bin}.
  \item To use these executables, you need to modify your system \inlsh{PATH} to
  include the library directories where DLLs are stored. To do this, navigate to
  \emph{Control Panel > System and Security > System}, select \emph{Advanced
    System Settings}, and in the \emph{Advanced} tab click the \emph{Environment
    Variables}. In the \emph{System Variables} box, select \emph{Path} and click
  \emph{Edit}. To the end of this list, add the \textbf{full paths} to
  directories:
  \begin{itemize}
    \item \inlsh{builds\textbackslash dist\textbackslash lib\textbackslash nektar++-\nekver}
    \item \inlsh{builds\textbackslash dist\textbackslash bin}
    \item Optionally, if you installed Boost from the binary packages,
    \inlsh{C:\textbackslash local\textbackslash boost\_1\_61\_0 \textbackslash lib}
  \end{itemize}
  \item To run the test suite, open a \textbf{new} command line window, change
  to the \inlsh{builds} directory, and then issue the command
  \begin{lstlisting}[style=BashInputStyle]
    ctest -C Release
  \end{lstlisting}
\end{enumerate}

\subsection{CMake Option Reference}
\label{s:installation:source:cmake}
This section describes the main configuration options which can be set when
building Nektar++. The default options should work on almost all systems, but
additional features (such as parallelisation and specialist libraries) can be
enabled if needed.

\subsubsection{Components}
The first set of options specify the components of the Nektar++ toolkit to
compile. Some options are dependent on others being enabled, so the available
options may change.

Components of the \nekpp package can be selected using the following options:
\begin{itemize}
    \item \inlsh{NEKTAR\_BUILD\_DEMOS} (Recommended)

    Compiles the demonstration programs. These are primarily used by the
    regression testing suite to verify the \nekpp library, but also provide an
    example of the basic usage of the framework.

    \item \inlsh{NEKTAR\_BUILD\_DOC}

    Compiles the Doxygen documentation for the code. This will be put in
    \begin{lstlisting}[style=BashInputStyle]
    $BUILDDIR/doxygen/html
    \end{lstlisting}

    \item \inlsh{NEKTAR\_BUILD\_LIBRARY} (Required)

    Compiles the Nektar++ framework libraries. This is required for all other
    options.

    \item \inlsh{NEKTAR\_BUILD\_SOLVERS} (Recommended)

    Compiles the solvers distributed with the \nekpp framework.

    If enabling \inlsh{NEKTAR\_BUILD\_SOLVERS}, individual solvers can be
    enabled or disabled. See Part~\ref{p:applications} for the list of available
    solvers. You can disable solvers which are not required to reduce
    compilation time. See the \inlsh{NEKTAR\_SOLVER\_X} option.

    \item \inlsh{NEKTAR\_BUILD\_TESTS} (Recommended)

    Compiles the testing program used to verify the \nekpp framework.

    \item \inlsh{NEKTAR\_BUILD\_TIMINGS}

    Compiles programs used for timing \nekpp operations.

    \item \inlsh{NEKTAR\_BUILD\_UNIT\_TESTS}

    Compiles tests for checking the core library functions.

    \item \inlsh{NEKTAR\_BUILD\_UTILITIES}

    Compiles utilities for pre- and post-processing simulation data.

    \item \inlsh{NEKTAR\_SOLVER\_X}

    Enabled compilation of the 'X' solver.
\end{itemize}

A number of ThirdParty libraries are required by \nekpp. There are
also optional libraries which provide additional functionality. These
can be selected using the following options:
\begin{itemize}
    \item \inlsh{NEKTAR\_USE\_ACML}

    Use the optimised BLAS library for AMD processors.

    \item \inlsh{NEKTAR\_USE\_ACCELERATE\_FRAMEWORK}

    Use the Mac Osx accelerate framework for BLAS and LAPACK
    methods. This option should only be required under in a Mac OSX
    environment.

    \item \inlsh{NEKTAR\_USE\_ARPACK}

    Build \nekpp with support for ARPACK. This provides routines used for
    linear stability analyses. Alternative Arnoldi algorithms are also
    implemented directly in \nekpp.

    \item \inlsh{NEKTAR\_USE\_BLAS\_LAPACK} (Required)

    Enables the use of Basic Linear Algebra Subroutines libraries for linear
    algebra operations.

    \item \inlsh{NEKTAR\_USE\_SYSTEM\_BLAS\_LAPACK} (Recommended)

    On Linux systems, use the default BLAS and LAPACK library on the system.
    This may not be the implementation offering the highest performance for your
    architecture, but it is the most likely to work without problem.

    \item \inlsh{NEKTAR\_USE\_CCM}

    Use the ccmio library provided with the Star-CCM package for
    reading ccm files. This option is required as part of NekMesh
    if you wish to convert a Star-CCM mesh into the Nektar format. It
    is possible to read a Tecplot plt file from Star-CCM but this
    output currently needs to be converted to ascii format using the
    Tecplot package.

    \item \inlsh{NEKTAR\_USE\_FFTW}

    Build \nekpp with support for FFTW for performing Fast Fourier Transforms
    (FFTs). This is used only when using domains with homogeneous coordinate
    directions.

    \item \inlsh{NEKTAR\_USE\_HDF5}

    Build \nekpp with support for HDF5. This enables input/output in the HDF5
    parallel file format, which can be very efficient for large numbers of
    processes. HDF5 output can be enabled by using a command-line option or
    in the \inlsh{SOLVERINFO} section of the XML file. This option requires 
    that \nekpp be built with MPI support with \inlsh{NEKTAR\_USE\_MPI} enabled
    and that HDF5 is compiled with MPI support.

    \item \inlsh{NEKTAR\_USE\_MKL}

    Use the Intel MKL library. This is typically available on cluster
    environments and should offer performance tuned for the specific cluster
    environment.

    \item \inlsh{NEKTAR\_USE\_MPI} (Recommended)

    Build Nektar++ with MPI parallelisation. This allows solvers to be run in
    serial or parallel.

    \item \inlsh{NEKTAR\_USE\_OPENBLAS}

    Use OpenBLAS for the BLAS library. OpenBLAS is based on the Goto BLAS
    implementation and generally offers better performance than a non-optimised
    system BLAS. However, the library must be installed on the system.

    \item \inlsh{NEKTAR\_USE\_PETSC}

    Build \nekpp with support for the PETSc package for solving linear systems.

    \item \inlsh{NEKTAR\_USE\_SCOTCH}

    Build \nekpp with support for the SCOTCH graph partitioning library. This
    provides an alternative mesh partitioning algorithm to METIS. However, METIS
    is still required as it is used by the multi-level static condensation
    algorithm.

    \item \inlsh{NEKTAR\_USE\_SMV}

    Build \nekpp with support for optimised sparse matrix-vector operations.

    \item \inlsh{NEKTAR\_USE\_VTK}

    Build \nekpp with support for VTK libraries. This is only needed for
    specialist utilities and is not needed for general use.

    \begin{notebox}
    The VTK libraries are not needed for converting the output of simulations to
    VTK format for visualization as this is handled internally.
    \end{notebox}
\end{itemize}

The \inlsh{THIRDPARTY\_BUILD\_X} options select which third-party libraries are
automatically built during the \nekpp build process. Below are the choices of X:
\begin{itemize}
    \item \inlsh{BOOST}

    The \emph{Boost} libraries are frequently provided by the operating system,
    so automatic compilation is not enabled by default. If you do not have
    Boost on your system, you can enable this to have Boost configured
    automatically.

    \item \inlsh{GSMPI}

    (MPI-only) Parallel communication library.

    \item \inlsh{LOKI}

    An implementation of a singleton.

    \item \inlsh{METIS}

    A graph partitioning library used for substructuring of matrices and mesh
    partitioning when \nekpp is run in parallel.

    \item \inlsh{PETSC}

    A package for the parallel solution of linear algebra systems.

    \item \inlsh{SCOTCH}

    An alternative graph partitioning library used for mesh partitioning when
    \nekpp is run in parallel.

    \item \inlsh{TINYXML}

    Library for reading and writing XML files.
\end{itemize}
