\chapter{XML Session File}
\label{s:xml}

The Nektar++ native file format is compliant with XML version 1.0. The root
element is NEKTAR which contains a number of other elements which describe
configuration for different aspects of the simulation. The required elements are
shown below:
\begin{lstlisting}[style=XMLStyle]
<NEKTAR>
  <GEOMETRY>
    ...
  </GEOMETRY>
  <EXPANSIONS>
    ...
  </EXPANSIONS>
  <CONDITIONS>
    ...
  </CONDITIONS>
  ...
</NEKTAR>
\end{lstlisting}
The different sub-elements can be split across multiple files, however each
file must have a top-level NEKTAR tag. For example, one might store the
geometry information separate from the remaining configuration in two separate
files as illustrated below:

\inlsh{geometry.xml}
\begin{lstlisting}[style=XMLStyle]
<NEKTAR>
  <GEOMETRY>
    ...
  </GEOMETRY>
</NEKTAR>
\end{lstlisting}

\inlsh{conditions.xml}
\begin{lstlisting}[style=XMLStyle]
<NEKTAR>
  <CONDITIONS>
    ...
  </CONDITIONS>
  <EXPANSIONS>
    ...
  </EXPANSIONS>
  ...
</NEKTAR>
\end{lstlisting}

\begin{notebox}
    When specifying multiple files, repeated XML sub-elements are not merged.
    The sub-elements from files appearing later in the list will, in general,
    override those elements from earlier files.
    \medskip
    
    For example, the \inlsh{NekMesh} utility will produce a default
    \inltt{EXPANSIONS} element and blank \inltt{CONDITIONS} element. Specifying
    a custom-written XML file containing these sections \emph{after} the
    file produced by \inlsh{NekMesh} will override these defaults.

    The exception to this rule is when an empty XML sub-element would override a
    non-empty XML sub-element. In this case the empty XML sub-element will be
    ignored. If the custom-written XML file containing \inltt{CONDITIONS} were
    specified before the file produced by \inlsh{NEKMESH}, the empty
    \inltt{CONDITIONS} tag in the latter file would be ignored.
\end{notebox}

\section{Geometry}
This section defines the mesh. It specifies a list of vertices, edges (in two or
three dimensions) and faces (in three dimensions) and how they connect to create
the elemental decomposition of the domain. It also defines a list of composites
which are used in the Expansions and Conditions sections of the file to describe
the polynomial expansions and impose boundary conditions.

The GEOMETRY section is structured as \begin{lstlisting}[style=XMLStyle]
<GEOMETRY DIM="2" SPACE="2">
  <VERTEX> ...
  </VERTEX> <EDGE> ...
  </EDGE> <FACE> ...
  </FACE> <ELEMENT> ...
  </ELEMENT> <CURVED> ...
  </CURVED> <COMPOSITE> ...
  </COMPOSITE> <DOMAIN> ... </DOMAIN>
</GEOMETRY>
\end{lstlisting}
It has two attributes:
\begin{itemize}
    \item \inltt{DIM} specifies the dimension of the expansion elements.
    \item \inltt{SPACE} specifies the dimension of the space in which the
    elements exist.
\end{itemize}

These attributes allow, for example, a two-dimensional surface to be embedded in
a three-dimensional space. The \inltt{FACES} section is only present when
\inltt{DIM}=3.
The \inltt{CURVED} section is only present if curved edges or faces are present
in the mesh.

\subsection{Vertices}

Vertices have three coordinates. Each has a unique vertex ID. They are defined
in the file within VERTEX subsection as follows:
\begin{lstlisting}[style=XMLStyle] <VERTEX>
  <V ID="0"> 0.0  0.0  0.0 </V> ...
</VERTEX>
\end{lstlisting}
VERTEX subsection has three optional attributes: \inltt{XSCALE},
\inltt{YSCALE} and \inltt{ZSCALE}. They specify scaling factors to
corresponding vertex coordinates.
For example, the following snippet

\begin{lstlisting}[style=XMLStyle] 
<VERTEX XSCALE="5">
  <V ID="0"> 0.0  0.0  0.0 </V> <V ID="1"> 1.0  2.0  0.0 </V>
</VERTEX>
\end{lstlisting}

defines two vertices with coordinates $(0.0,0.0,0.0), (5.0,2.0,0.0)$. Values of
\inltt{XSCALE}, \inltt{YSCALE} and \inltt{ZSCALE} attributes can be arbitrary
analytic expressions depending on pre-defined constants, parameters defined earlier in the XML file and
mathematical operations/functions of the latter. If omitted, default scaling
factors 1.0 are assumed.

\subsection{Edges}

Edges are defined by two vertices. Each edge has a unique edge ID. They are
defined in the file with a line of the form 

\begin{lstlisting}[style=XMLStyle]
<E ID="0"> 0 1 </E>
\end{lstlisting}

\subsection{Faces}

Faces are defined by three or more edges. Each face has a unique face ID. They
are defined in the file with a line of the form

\begin{lstlisting}[style=XMLStyle]
<T ID="0"> 0 1 2 </T>
<Q ID="1"> 3 4 5 6 </Q>
\end{lstlisting}

The choice of tag specified (T or Q), and thus the number of edges specified depends on the geometry of the face (triangle or quadrilateral).

\subsection{Element}

Elements define the top-level geometric entities in the mesh. Their definition depends upon the dimension of the expansion. For two-dimensional expansions, an element is defined by a sequence of three or four edges. For three-dimensional expansions, the element is defined by a list of faces. Elements are defined in the file with a line of the form
\begin{lstlisting}[style=XMLStyle]
<T ID="0"> 0 1 2 </T>
<H ID="1"> 3 4 5 6 7 8 </H>
\end{lstlisting}
Again, the choice of tag specified depends upon the geometry of the element. The element tags are:

\begin{itemize}
    \item \inltt{S} Segment
    \item \inltt{T} Triangle
    \item \inltt{Q} Quadrilateral
    \item \inltt{A} Tetrahedron
    \item \inltt{P} Pyramid
    \item \inltt{R} Prism
    \item \inltt{H} Hexahedron
\end{itemize}


\subsection{Curved Edges and Faces}

For mesh elements with curved edges and/or curved faces, a separate entry is used to describe the control points for the curve. Each curve has a unique curve ID and is associated with a predefined edge or face. The total number of points in the curve (including end points) and their distribution is also included as attributes. The control points are listed in order, each specified by three coordinates. Curved edges are defined in the file with a line of the form
\begin{lstlisting}[style=XMLStyle]
<E ID="3" EDGEID="7" TYPE="PolyEvenlySpaced" NUMPOINTS="3">
    0.0  0.0  0.0    0.5  0.5  0.0    1.0  0.0  0.0
</E>
\end{lstlisting}

\subsection{Composites}

Composites define collections of elements, faces or edges. Each has a unique composite ID associated with it. All components of a composite entry must be of the same type. The syntax allows components to be listed individually or using ranges. Examples include
\begin{lstlisting}[style=XMLStyle]
<C ID="0"> T[0-862] </C>
<C ID="1"> E[68,69,70,71] </C>
\end{lstlisting}

\subsection{Domain}

This tag specifies composites which describe the entire problem domain. It has the form of
\begin{lstlisting}[style=XMLStyle]
<DOMAIN> C[0] </DOMAIN>
\end{lstlisting}


\section{Expansions}
This section defines the polynomial expansions used on each of the defined
geometric composites. Expansion entries specify the number of modes, the basis
type. The short-hand version has the following form

\begin{lstlisting}[style=XMLStyle]
<E COMPOSITE="C[0]" NUMMODES="5" FIELDS="u" TYPE="MODIFIED" />
\end{lstlisting}

or, if we have more then one variable we can apply the same basis to all using

\begin{lstlisting}[style=XMLStyle]
<E COMPOSITE="C[0]" NUMMODES="5" FIELDS="u,v,p" TYPE="MODIFIED" />
\end{lstlisting}

The expansion basis can also be specified in detail as a combination of
one-dimensional bases, and thus the user is able to, for example, increase the
quadrature order. For tet elements this takes the form:

\begin{lstlisting}[style=XMLStyle]
<E COMPOSITE="C[0]" 
   BASISTYPE="Modified_A,Modified_B,Modified_C" 
   NUMMODES="3,3,3"
   POINTSTYPE="GaussLobattoLegendre,GaussRadauMAlpha1Beta0,GaussRadauMAlpha2Beta0"
   NUMPOINTS="4,3,3"
   FIELDS="u" />
\end{lstlisting}

and for prism elements:

\begin{lstlisting}[style=XMLStyle]
<E COMPOSITE="C[1]" 
   BASISTYPE="Modified_A,Modified_A,Modified_B" 
   NUMMODES="3,3,3"
   POINTSTYPE="GaussLobattoLegendre,GaussLobattoLegendre,GaussRadauMAlpha1Beta0"
   NUMPOINTS="4,4,3"
   FIELDS="u" />
\end{lstlisting}



\section{Conditions}
The final section of the file defines parameters and boundary conditions which
define the nature of the problem to be solved. These are enclosed in the
\inltt{CONDITIONS} tag.

\subsection{Parameters}

Parameters may be required by a particular solver (for instance time-integration
parameters or solver-specific parameters), or arbitrary and only used within the
context of the session file (e.g. parameters in the definition of an initial
condition). All parameters are enclosed in the \inltt{PARAMETERS} XML element.

\begin{lstlisting}[style=XMLStyle] 
<PARAMETERS>
    ...
</PARAMETERS>
\end{lstlisting}

A parameter may be of integer or real type and may reference other parameters
defined previous to it. It is expressed in the file as

\begin{lstlisting}[style=XMLStyle]
<P> [PARAMETER NAME] = [PARAMETER VALUE] </P>
\end{lstlisting}

For example,

\begin{lstlisting}[style=XMLStyle]
<P> NumSteps = 1000              </P>
<P> TimeStep = 0.01              </P>
<P> FinTime  = NumSteps*TimeStep </P>
\end{lstlisting}

\subsection{Solver Information}

These specify properties to define the actions specific to solvers, typically
including the equation to solve, the projection type and the method of time
integration. The property/value pairs are specified as XML attributes. For
example, 
\begin{lstlisting}[style=XMLStyle] 
<SOLVERINFO>
  <I PROPERTY="EQTYPE"                VALUE="UnsteadyAdvection"    /> 
  <I PROPERTY="Projection"            VALUE="Continuous"           /> 
  <I PROPERTY="TimeIntegrationMethod" VALUE="ClassicalRungeKutta4" />
</SOLVERINFO>
\end{lstlisting}

The list of available solvers in Nektar++ can be found in
Part~\ref{p:applications}.

\subsubsection{Drivers}
Drivers are defined under the \inltt{CONDITIONS} section as properties of the 
\inltt{SOLVERINFO} XML element. The role of a driver is to manage the solver 
execution from an upper level. 

The default driver is called \inltt{Standard} and executes the following steps:
\begin{enumerate}
\item Prints out on screen a summary of all the conditions defined in the input file.
\item Sets up the initial and boundary conditions.
\item Calls the solver defined by \inltt{SolverType}  in the \inltt{SOLVERINFO} XML element.
\item Writes the results in the output (.fld) file.
\end{enumerate}

In the following example, the driver \inltt{Standard} is used to manage the 
execution of the incompressible Navier-Stokes equations:
\begin{lstlisting}[style=XMLStyle]
<SOLVERINFO>
    <I PROPERTY="EQTYPE" VALUE="UnsteadyNavierStokes" />
    <I PROPERTY="SolverType" VALUE="VelocityCorrectionScheme" />
    <I PROPERTY="Projection" VALUE="Galerkin" />
    <I PROPERTY="TimeIntegrationMethod" VALUE="IMEXOrder2" />
    <I PROPERTY="Driver" VALUE="Standard" />
</SOLVERINFO>
\end{lstlisting}

If no driver is specified in the session file, the driver \inltt{Standard} is 
called by default. Other drivers can be used and are typically focused on
specific applications. As described in Sec.
\ref{SectionIncNS_SolverInfo} and  \ref{SectionIncNS_SolverInfo_Stab}, 
the other possibilities are:
\begin{itemize}
\item \inltt{ModifiedArnoldi}  - computes of the leading eigenvalues and 
eigenmodes using modified Arnoldi method.
\item \inltt{Arpack} - computes of eigenvalues/eigenmodes using Implicitly 
Restarted Arnoldi Method (ARPACK).
\item \inltt{SteadyState} - uses the Selective Frequency Damping method 
(see Sec. \ref{SectionSFD}) to obtain a steady-state solution of the 
Navier-Stokes equations (compressible or incompressible).
\end{itemize}


\subsection{Variables}

These define the number (and name) of solution variables. Each variable is
prescribed a unique ID. For example a two-dimensional flow simulation may define
the velocity variables using

\begin{lstlisting}[style=XMLStyle]
<VARIABLES>
  <V ID="0"> u </V>
  <V ID="1"> v </V>
</VARIABLES>
\end{lstlisting}

\subsection{Global System Solution Information}

This section allows you to specify the global system solution parameters which
is particularly useful when using an iterative solver. An example of this
section is as follows:

\begin{lstlisting}[style=XMLStyle]
<GLOBALSYSSOLNINFO>
  <V VAR="u,v,w">
    <I PROPERTY="GlobalSysSoln"             VALUE="IterativeStaticCond" />
    <I PROPERTY="Preconditioner"            VALUE="LowEnergyBlock"/>
    <I PROPERTY="IterativeSolverTolerance"  VALUE="1e-8"/>
  </V>
  <V VAR="p">
    <I PROPERTY="GlobalSysSoln"             VALUE="IterativeStaticCond" />
    <I PROPERTY="Preconditioner"     VALUE="FullLinearSpaceWithLowEnergyBlock"/>
    <I PROPERTY="IterativeSolverTolerance"  VALUE="1e-6"/>
  </V>
</GLOBALSYSSOLNINFO>
\end{lstlisting}

The above section specifies that the global solution system for the variables
"u,v,w" should use the iIerativeStaticCond approach with the LowEnergyBlock
preconditioned and an iterative tolerance of 1e-6.  Where as the variable "p"
which also is solved with the IterativeStaticCond approach should use the
FullLinearSpaceWithLowEnergyBlock and an iterative tolerance of 1e-8.

Other parameters which can be specified include SuccessiveRHS. 

The parameters in this section override those specified in the Parameters section. 

\subsection{Boundary Regions and Conditions}

Boundary conditions are defined by two XML elements. The first defines the
boundary regions in the domain in terms of composite entities from the
\inltt{GEOMETRY} section of the file. Each boundary region has a unique ID and
are defined as, 
\begin{lstlisting}[style=XMLStyle]
<BOUNDARYREGIONS>
    <B ID=[id]> [composite-list] </B>
    ...
</BOUNDARYREGIONS>
\end{lstlisting}
For example,
\begin{lstlisting}[style=XMLStyle]
<BOUNDARYREGIONS>
  <B ID="0"> C[2] </B>
  <B ID="1"> C[3] </B>
</BOUNDARYREGIONS>
\end{lstlisting}

The second XML element defines, for each variable, the condition to impose on
each boundary region, and has the form,
\begin{lstlisting}[style=XMLStyle]
<BOUNDARYCONDITIONS>
    <REGION REF="[regionID]">
      <[type1] VAR="[variable1]" VALUE="[expression1]" />
      ...
      <[typeN] VAR="[variableN]" VALUE="[expressionN]" />
    </REGION>
    ...
</BOUNDARYCONDITIONS>
\end{lstlisting}
There should be precisely one \inltt{REGION} entry for each \inltt{B} entry
defined in the \inltt{BOUNDARYREGION} section above. For example, to impose a
Dirichlet condition on both variables for a domain with a single region, 
\begin{lstlisting}[style=XMLStyle] 
<BOUNDARYCONDITIONS>
  <REGION REF="0">
    <D VAR="u" VALUE="sin(PI*x)*cos(PI*y)" /> 
    <D VAR="v" VALUE="sin(PI*x)*cos(PI*y)" />
  </REGION>
</BOUNDARYCONDITIONS>
\end{lstlisting}
Boundary condition specifications may refer to any parameters defined in the
session file. The \inltt{REF} attribute corresponds to a defined boundary
region. The tag used for each variable specifies the type of boundary condition
to enforce.

\subsubsection{Dirichlet (essential) condition}
Dirichlet conditions are specified with the \inltt{D} tag.

\begin{tabular}{llll}
Projection & Homogeneous support & Time-dependent support & Dimensions \\
\toprule
CG & Yes & Yes & 1D, 2D and 3D \\
DG & Yes & Yes & 1D, 2D and 3D \\
HDG& Yes & Yes & 1D, 2D and 3D
\end{tabular}

Example:
\begin{lstlisting}[style=XMLStyle]
<!-- homogeneous condition -->
<D VAR="u" VALUE="0" />
<!-- inhomogeneous condition -->
<D VAR="u" VALUE="x^2+y^2+z^2" />
<!-- time-dependent condition -->
<D VAR="u" USERDEFINEDTYPE="TimeDependent" VALUE="x+t" />
\end{lstlisting}

\subsubsection{Neumann (natural) condition}
Neumann conditions are specified with the \inltt{N} tag.

\begin{tabular}{llll}
Projection & Homogeneous support & Time-dependent support & Dimensions \\
\toprule
CG & Yes & Yes & 1D, 2D and 3D \\
DG & No  & No  & 1D, 2D and 3D \\
HDG & ? & ? & ?
\end{tabular}

Example:
\begin{lstlisting}[style=XMLStyle]
<!-- homogeneous condition -->
<N VAR="u" VALUE="0" />
<!-- inhomogeneous condition -->
<N VAR="u" VALUE="x^2+y^2+z^2" />
<!-- time-dependent condition -->
<N VAR="u" USERDEFINEDTYPE="TimeDependent" VALUE="x+t" />
<!-- high-order pressure boundary condition (for IncNavierStokesSolver) -->
<N VAR="u" USERDEFINEDTYPE="H" VALUE="0" />
\end{lstlisting}

\subsubsection{Periodic condition}
Periodic conditions are specified with the \inltt{P} tag.

\begin{tabular}{lll}
Projection & Homogeneous support & Dimensions \\
\toprule
CG  & Yes & 1D, 2D and 3D \\
DG  & No  & 2D and 3D
\end{tabular}

Example:
\begin{lstlisting}[style=XMLStyle]
<BOUNDARYREGIONS>
  <B ID="0"> C[1] </B>
  <B ID="1"> C[2] </B>
</BOUNDARYREGIONS>

<BOUNDARYCONDITIONS>
  <REGION REF="0">
    <P VAR="u" VALUE="[1]" />
  </REGION>
  <REGION REF="1">
    <P VAR="u" VALUE="[0]" />
  </REGION>
</BOUNDARYCONDITIONS>
\end{lstlisting}

Periodic boundary conditions are specified in a significantly different form to
other conditions. The \inltt{VALUE} property is used to specify which
\inltt{BOUNDARYREGION} is periodic with the current region in square brackets.

Caveats:
\begin{itemize}
\item A periodic condition must be set for '''both''' boundary regions; simply
 specifying a condition for region 0 or 1 in the above example is not enough.
\item The order of the elements inside the composites defining periodic
boundaries is important. For example, if `C[0]` above is defined as edge IDs 
`{0,5,4,3}` and `C[1]` as `{7,12,2,1}` then edge 0 is periodic with edge 7, 5 
with 12, and so on.
\item For the above reason, the composites must also therefore be of the same
size.
\item In three dimensions, care must be taken to correctly align triangular
faces which are intended to be periodic. The top (degenerate) vertex should be 
aligned so that, if the faces were connected, it would lie at the same point on 
both triangles.
\end{itemize}

\subsubsection{Time-dependent boundary conditions}
Time-dependent boundary conditions may be specified through setting the
\inltt{USERDEFINEDTYPE} attribute and using the parameter \inltt{t} where the
current time is required. For example,
\begin{lstlisting}[style=XMLStyle]
<D VAR="u" USERDEFINEDTYPE="TimeDependent" VALUE="sin(PI*(x-t))" />
\end{lstlisting}

\subsubsection{Boundary conditions from file}
Boundary conditions can also be loaded from file. The following example is from
the Incompressible Navier-Stokes solver,
\begin{lstlisting}[style=XMLStyle]
<REGION REF="1">
  <D VAR="u" FILE="Test_ChanFlow2D_bcsfromfiles_u_1.bc" />
  <D VAR="v" VALUE="0" />
  <N VAR="p" USERDEFINEDTYPE="H" VALUE="0" />
</REGION>
\end{lstlisting}

Boundary conditions can also be loaded simultaneously from a file and from an analytic expression. For example in the scenario where a spatial boundary condition is read from a file, but needs to be modulated by a time-dependent analytic expression:
\begin{lstlisting}[style=XMLStyle]
<REGION REF="1">
  <D VAR="u"	VALUE="USERDEFINEDTYPE="TimeDependent" VALUE="sin(PI*(x-t)) 
  FILE="bcsfromfiles_u_1.bc" />
</REGION>
\end{lstlisting}

In the case where both \inltt{VALUE} and \inltt{FILE} are specified, the values in both are multiplied together to give the final value for the boundary condition. 

\subsection{Functions}

Finally, multi-variable functions such as initial conditions and analytic
solutions may be specified for use in, or comparison with, simulations. These
may be specified using expressions (\inltt{<E>}) or imported from a file
(\inltt{<F>}) using the Nektar++ FLD file format

\begin{lstlisting}[style=XMLStyle]
<FUNCTION NAME="ExactSolution">
  <E VAR="u" VALUE="sin(PI*x-advx*t))*cos(PI*(y-advy*t))" />
</FUNCTION>
<FUNCTION NAME="InitialConditions">
  <F VAR="u" FILE="session.rst" />
</FUNCTION>
\end{lstlisting}

A restart file is a solution file (in other words an .fld renamed as .rst) where
the field data is specified. The expansion order used to generate the .rst file
must be the same as that for the simulation.
.pts files contain scattered point data which needs to be interpolated to the field.
For further information on the file format and the different interpolation schemes, see
section~\ref{s:utilities:fieldconvert:sub:interppointdatatofld}.
All filenames must be specified relative to the location of the .xml file.

With the additional argument \inltt{TIMEDEPENDENT="1"}, different files can be
loaded for each timestep. The filenames are defined using
\href{http://www.boost.org/doc/libs/1_56_0/libs/format/doc/format.html#syntax}{boost::format syntax}
where the step time is used as variable. For example, the function
\inltt{Baseflow} would load the files \inltt{U0V0\_1.00000000E-05.fld},
\inltt{U0V0\_2.00000000E-05.fld} and so on.

\begin{lstlisting}[style=XMLStyle]
<FUNCTION NAME="Baseflow">
       <F VAR="U0,V0" TIMEDEPENDENT="1"FILE="U0V0_%14.8E.fld"/>
</FUNCTION>
\end{lstlisting}

For .pts files, the time consuming computation of interpolation weights in only
performed for the first timestep. The weights are stored and reused in all subsequent steps, 
which is why all consecutive .pts files must use the same ordering, number and location of
data points.

Other examples of this input feature can be the insertion of a forcing term,

\begin{lstlisting}[style=XMLStyle]
<FUNCTION NAME="BodyForce">
  <E VAR="u" VALUE="0" />
  <E VAR="v" VALUE="0" />
</FUNCTION>
<FUNCTION NAME="Forcing">
  <E VAR="u" VALUE="-(Lambda + 2*PI*PI)*sin(PI*x)*sin(PI*y)" />
</FUNCTION>
\end{lstlisting}

or of a linear advection term

\begin{lstlisting}[style=XMLStyle]
<FUNCTION NAME="AdvectionVelocity">
  <E VAR="Vx" VALUE="1.0" />
  <E VAR="Vy" VALUE="1.0" />
  <E VAR="Vz" VALUE="1.0" />
</FUNCTION>
\end{lstlisting}

\subsubsection{Remapping variable names}

Note that it is sometimes the case that the variables being used in the solver
do not match those saved in the FLD file. For example, if one runs a
three-dimensional incompressible Navier-Stokes simulation, this produces an FLD
file with the variables \inltt{u}, \inltt{v}, \inltt{w} and \inltt{p}. If we
wanted to use this velocity field as input for an advection velocity, the
advection-diffusion-reaction solver expects the variables \inltt{Vx}, \inltt{Vy}
and \inltt{Vz}.

We can manually specify this mapping by adding a colon to the

\begin{lstlisting}[style=XMLStyle]
<FUNCTION NAME="AdvectionVelocity">
  <F VAR="Vx,Vy,Vz" FILE="file.fld:u,v,w" />
</FUNCTION>
\end{lstlisting}

There are some caveats with this syntax:

\begin{itemize}
  \item You must specify the same number of fields for both the variable, and
  after the colon. For example, the following is not valid.
  \begin{lstlisting}[style=XMLStyle,gobble=4]
    <FUNCTION NAME="AdvectionVelocity">
      <F VAR="Vx,Vy,Vz" FILE="file.fld:u" />
    </FUNCTION>\end{lstlisting}
  \item This syntax is not valid with the wildcard operator \inltt{*}, so one
  cannot write for example:
  \begin{lstlisting}[style=XMLStyle,gobble=4]
    <FUNCTION NAME="AdvectionVelocity">
      <F VAR="*" FILE="file.fld:u,v,w" />
    </FUNCTION>
  \end{lstlisting}
\end{itemize}

\subsubsection{Time-dependent file-based functions}

With the additional argument \inltt{TIMEDEPENDENT="1"}, different files can be
loaded for each timestep. The filenames are defined using
\href{http://www.boost.org/doc/libs/1_56_0/libs/format/doc/format.html#syntax}{boost::format
  syntax} where the step time is used as variable. For example, the function
\inltt{Baseflow} would load the files \inltt{U0V0\_1.00000000E-05.fld},
\inltt{U0V0\_2.00000000E-05.fld} and so on.

\begin{lstlisting}[style=XMLStyle]
<FUNCTION NAME="Baseflow">
  <F VAR="U0,V0" TIMEDEPENDENT="1" FILE="U0V0_%14.8R.fld" />
</FUNCTION>
\end{lstlisting}

Section~\ref{sec:xml:analytic-expressions} provides the list of acceptable
mathematical functions and other related technical details.

\subsection{Quasi-3D approach}

To generate a Quasi-3D appraoch with Nektar++ we only need to create a 2D or a
1D mesh, as reported above, and then specify the parameters to extend the
problem to a 3D case. For a 2D spectral/hp element problem, we have a 2D mesh
and along with the parameters we need to define the problem (i.e. equation type,
boundary conditions, etc.). The only thing we need to do, to extend it to a
Quasi-3D approach, is to specify some additional parameters which characterise
the harmonic expansion in the third direction. First we need to specify in the
solver information section that that the problem will be extended to have one
homogeneouns dimension; here an example

\begin{lstlisting}[style=XMLStyle]
<SOLVERINFO>
  <I PROPERTY="SolverType"            VALUE="VelocityCorrectionScheme"/>
  <I PROPERTY="EQTYPE"                VALUE="UnsteadyNavierStokes"/>
  <I PROPERTY="AdvectionForm"         VALUE="Convective"/>
  <I PROPERTY="Projection"            VALUE="Galerkin"/>
  <I PROPERTY="TimeIntegrationMethod" VALUE="IMEXOrder2"/>
  <I PROPERTY="HOMOGENEOUS"           VALUE="1D"/>
</SOLVERINFO>
\end{lstlisting}

then we need to specify the parameters which define the 1D harmonic expanson
along the z-axis, namely the homogeneous length (LZ) and the number of modes in
the homogeneous direction (HomModesZ). HomModesZ corresponds also to the number
of quadrature points in the homogenous direction, hence on the number of 2D
planes discretized with a specral/hp element method.

\begin{lstlisting}[style=XMLStyle]
<PARAMETERS>
  <P> TimeStep      = 0.001   </P>
  <P> NumSteps      = 1000    </P>
  <P> IO_CheckSteps = 100     </P>
  <P> IO_InfoSteps  = 10      </P>
  <P> Kinvis        = 0.025   </P>
  <P> HomModesZ     = 4       </P>
  <P> LZ            = 1.0     </P>
</PARAMETERS>
\end{lstlisting}

In case we want to create a Quasi-3D approach starting form a 1D spectral/hp
element mesh, the procedure is the same, but we need to specify the parameters
for two harmonic directions (in Y and Z direction). For Example,

\begin{lstlisting}[style=XMLStyle]
<SOLVERINFO>
  <I PROPERTY="EQTYPE"                VALUE="UnsteadyAdvectionDiffusion" />
  <I PROPERTY="Projection"            VALUE="Continuous"/>
  <I PROPERTY="HOMOGENEOUS"           VALUE="2D"/>
  <I PROPERTY="DiffusionAdvancement"  VALUE="Implicit"/>
  <I PROPERTY="AdvectionAdvancement"  VALUE="Explicit"/>
  <I PROPERTY="TimeIntegrationMethod" VALUE="IMEXOrder2"/>
</SOLVERINFO>
<PARAMETERS>
  <P> TimeStep      = 0.001 </P>
  <P> NumSteps      = 200   </P>
  <P> IO_CheckSteps = 200   </P>
  <P> IO_InfoSteps  = 10    </P>
  <P> wavefreq      = PI    </P>
  <P> epsilon       = 1.0   </P>
  <P> Lambda        = 1.0   </P>
  <P> HomModesY     = 10    </P>
  <P> LY            = 6.5   </P>
  <P> HomModesZ     = 6     </P>
  <P> LZ            = 2.0   </P>
</PARAMETERS>
\end{lstlisting}

By default the opeartions associated with the harmonic expansions are performed
with the Matix-Vector-Multiplication (MVM) defined inside the code. The Fast
Fourier Transofrm (FFT) can be used to speed up the operations (if the FFTW
library has been compiled in ThirdParty, see the compilation instructions). To
use the FFT routines we need just to insert a flag in the solver information as
below:

\begin{lstlisting}[style=XMLStyle]
<SOLVERINFO>
  <I PROPERTY="EQTYPE"                VALUE="UnsteadyAdvectionDiffusion" />
  <I PROPERTY="Projection"            VALUE="Continuous"/>
  <I PROPERTY="HOMOGENEOUS"           VALUE="2D"/>
  <I PROPERTY="DiffusionAdvancement"  VALUE="Implicit"/>
  <I PROPERTY="AdvectionAdvancement"  VALUE="Explicit"/>
  <I PROPERTY="TimeIntegrationMethod" VALUE="IMEXOrder2"/>
  <I PROPERTY="USEFFT"                VALUE="FFTW"/>
</SOLVERINFO>
\end{lstlisting}

The number of homogenenous modes has to be even. The Quasi-3D apporach can be
created starting from a 2D mesh and adding one homogenous expansion or starting
form a 1D mesh and adding two homogeneous expansions. Not other options
available. In case of a 1D homogeneous extension, the homogeneous direction will
be the z-axis. In case of a 2D homogeneous extension, the homogeneous directions
will be the y-axis and the z-axis.

%%% Local Variables:
%%% mode: latex
%%% TeX-master: "../user-guide"
%%% End:


\section{Filters}

Filters are a method for calculating a variety of useful quantities from the
field variables as the solution evolves in time, such as time-averaged fields
and extracting the field variables at certain points inside the domain. Each
filter is defined in a \inltt{FILTER} tag inside a \inltt{FILTERS} block which
lies in the main \inltt{NEKTAR} tag. In this section we give an overview of the
modules currently available and how to set up these filters in the session file.

Here is an example \inltt{FILTER}:

\begin{lstlisting}[style=XMLStyle,gobble=2]
  <FILTER TYPE="FilterName">
      <PARAM NAME="Param1"> Value1 </PARAM>
      <PARAM NAME="Param2"> Value2 </PARAM>
  </FILTER>
\end{lstlisting}

A filter has a name -- in this case, \inltt{FilterName} -- together with
parameters which are set to user-defined values. Each filter expects different
parameter inputs, although where functionality is similar, the same parameter
names are shared between filter types for consistency. Numerical filter
parameters may be expressions and so may include session parameters defined in
the \inltt{PARAMETERS} section.

In the following we document the filters implemented. Note that some filters are
solver-specific and will therefore only work for a given subset of the available
solvers.

\subsection{FieldConvert checkpoints}

\begin{notebox}
  This filter is still at an experimental stage. Not all modules and options
  from FieldConvert are supported.
\end{notebox}

This filter applies a sequence of FieldConvert modules to the solution, 
writing an output file. An output is produced at the end of the simulation into
\inltt{session\_fc.fld}, or alternatively every $M$ timesteps as defined by the
user, into a sequence of files \inltt{session\_*\_fc.fld}, where \inltt{*} is
replaced by a counter.

The following parameters are supported:

\begin{center}
  \begin{tabularx}{0.99\textwidth}{lllX}
    \toprule
    \textbf{Option name} & \textbf{Required} & \textbf{Default} & 
    \textbf{Description} \\
    \midrule
    \inltt{OutputFile}      & \xmark   & \texttt{session.fld} &
    Output filename. If no extension is provided, it is assumed as .fld\\
    \inltt{OutputFrequency} & \xmark   & \texttt{NumSteps} &
    Number of timesteps after which output is written, $M$.\\
    \inltt{Modules} & \xmark   &  &
    FieldConvert modules to run, separated by a white space.\\
    \bottomrule
  \end{tabularx}
\end{center}

As an example, consider:

\begin{lstlisting}[style=XMLStyle,gobble=2]
  <FILTER TYPE="FieldConvert">
      <PARAM NAME="OutputFile">MyFile.vtu</PARAM>
      <PARAM NAME="OutputFrequency">100</PARAM>
      <PARAM NAME="Modules"> vorticity isocontour:fieldid=0:fieldvalue=0.1 </PARAM>
  </FILTER>
\end{lstlisting}

This will create a sequence of files named \inltt{MyFile\_*\_fc.vtu} containing isocontours. 
The result will be output every 100 time steps. Output directly to 
\inltt{.vtu} or \inltt{.dat} is currently only supported for isocontours.
In other cases, the output should be a \inltt{.fld} file.

\subsection{Time-averaged fields}

This filter computes time-averaged fields for each variable defined in the
session file. Time averages are computed by either taking a snapshot of the
field every timestep, or alternatively at a user-defined number of timesteps
$N$. An output is produced at the end of the simulation into
\inltt{session\_avg.fld}, or alternatively every $M$ timesteps as defined by the
user, into a sequence of files \inltt{session\_*\_avg.fld}, where \inltt{*} is
replaced by a counter. This latter option can be useful to observe statistical
convergence rates of the averaged variables.

This filter is derived from FieldConvert filter, and therefore support all parameters
available in that case. The following additional parameter is supported:

\begin{center}
  \begin{tabularx}{0.99\textwidth}{lllX}
    \toprule
    \textbf{Option name} & \textbf{Required} & \textbf{Default} & 
    \textbf{Description} \\
    \midrule
    \inltt{SampleFrequency} & \xmark   & 1 &
    Number of timesteps at which the average is calculated, $N$.\\
    \inltt{RestartFile} & \xmark   &   &
    Restart file used as initial average.
    If no extension is provided, it is assumed as .fld\\
  \end{tabularx}
\end{center}

As an example, consider:

\begin{lstlisting}[style=XMLStyle,gobble=2]
  <FILTER TYPE="AverageFields">
      <PARAM NAME="OutputFile">MyAverageField</PARAM>
      <PARAM NAME="RestartFile">MyRestartAvg.fld</PARAM>
      <PARAM NAME="OutputFrequency">100</PARAM>
      <PARAM NAME="SampleFrequency"> 10 </PARAM>		
  </FILTER>
\end{lstlisting}

This will create a file named \inltt{MyAverageField.fld} averaging the
instantaneous fields every 10 time steps. The averaged field is however only
output every 100 time steps.

\subsection{Moving average of fields}

This filter computes the exponential moving average (in time) of
fields for each variable defined in the session file. The moving average 
is defined as:
\[
\bar{u}_n = \alpha u_n + (1 - \alpha)\bar{u}_{n-1}
\]
with $0 < \alpha < 1$ and $\bar{u}_1 = u_1$.

The same parameters of the time-average filter are supported, with the output file
in the form \inltt{session\_*\_movAvg.fld}. In addition,
either $\alpha$ or the time-constant $\tau$ must be defined. They are related by:
\[
\alpha = \frac{t_s}{\tau + t_s}
\]
where $t_s$ is the time interval between consecutive samples.

As an example, consider:

\begin{lstlisting}[style=XMLStyle,gobble=2]
  <FILTER TYPE="MovingAverage">
      <PARAM NAME="OutputFile">MyMovingAverage</PARAM>
      <PARAM NAME="OutputFrequency">100</PARAM>
      <PARAM NAME="SampleFrequency"> 10 </PARAM>
      <PARAM NAME="tau"> 0.1 </PARAM>
  </FILTER>
\end{lstlisting}

This will create a file named \inltt{MyMovingAverage\_movAvg.fld} with a moving average
sampled every 10 time steps. The averaged field is however only
output every 100 time steps.

\subsection{Reynolds stresses}

\begin{notebox}
  This filter is only supported for the incompressible Navier-Stokes solver.
\end{notebox}

This filter is an extended version of the time-average filter. It outputs
not only the time-average of the fields, but also the Reynolds stresses.
The same parameters supported in the time-average case can be used,
for example:

\begin{lstlisting}[style=XMLStyle,gobble=2]
  <FILTER TYPE="ReynoldsStresses">
      <PARAM NAME="OutputFile">MyAverageField</PARAM>
      <PARAM NAME="RestartFile">MyAverageRst.fld</PARAM>
      <PARAM NAME="OutputFrequency">100</PARAM>
      <PARAM NAME="SampleFrequency"> 10 </PARAM>
  </FILTER>
\end{lstlisting}

By default, this filter uses a simple average. Optionally, an exponential
moving average can be used, in which case the output contains the moving
averages and the Reynolds stresses calculated based on them. For example:

\begin{lstlisting}[style=XMLStyle,gobble=2]
  <FILTER TYPE="ReynoldsStresses">
      <PARAM NAME="OutputFile">MyAverageField</PARAM>
      <PARAM NAME="MovingAverage">true</PARAM>
      <PARAM NAME="OutputFrequency">100</PARAM>
      <PARAM NAME="SampleFrequency"> 10 </PARAM>
      <PARAM NAME="alpha"> 0.01 </PARAM>
  </FILTER>
\end{lstlisting}

\subsection{Checkpoint fields}
 
The checkpoint filter writes a checkpoint file, containing the instantaneous
state of the solution fields at at given timestep. This can subsequently be used
for restarting the simulation or examining time-dependent behaviour. This
produces a sequence of files, by default named \inltt{session\_*.chk}, where
\inltt{*} is replaced by a counter. The initial condition is written to
\inltt{session\_0.chk}.

\begin{notebox}
  This functionality is equivalent to setting the \inltt{IO\_CheckSteps}
  parameter in the session file.
\end{notebox}

The following parameters are supported:

\begin{center}
  \begin{tabularx}{0.99\textwidth}{lllX}
    \toprule
    \textbf{Option name} & \textbf{Required} & \textbf{Default} & 
    \textbf{Description} \\
    \midrule
    \inltt{OutputFile}      & \xmark   & \texttt{session} &
    Prefix of the output filename to which the checkpoints are written.\\
    \inltt{OutputFrequency} & \cmark   & - &
    Number of timesteps after which output is written.\\
    \bottomrule
  \end{tabularx}
\end{center}

For example, to output the fields every 100 timesteps we can specify:

\begin{lstlisting}[style=XMLStyle,gobble=2]
  <FILTER TYPE="Checkpoint">
      <PARAM NAME="OutputFile">IntermediateFields</PARAM>
      <PARAM NAME="OutputFrequency">100</PARAM>
  </FILTER>
\end{lstlisting}
 
\subsection{History points}

The history points filter can be used to evaluate the value of the fields in
specific points of the domain as the solution evolves in time. By default this 
produces a file called \inltt{session.his}. For each timestep, and then each 
history point, a line is output containing the current solution time, followed 
by the value of each of the field variables. Commented lines are created at the
top of the file containing the location of the history points and the order of 
the variables.

The following parameters are supported:

\begin{center}
  \begin{tabularx}{0.99\textwidth}{lllX}
    \toprule
    \textbf{Option name} & \textbf{Required} & \textbf{Default} & 
    \textbf{Description} \\
    \midrule
    \inltt{OutputFile}      & \xmark   & \texttt{session} &
    Prefix of the output filename to which the checkpoints are written.\\
    \inltt{OutputFrequency} & \xmark   & 1 &
    Number of timesteps after which output is written.\\
    \inltt{OutputPlane}     & \xmark   & 0 &
    If the simulation is homogeneous, the plane on which to evaluate the 
    history point. (No Fourier interpolation is currently implemented.)\\
    \inltt{Points      }    & \cmark   & - &
    A list of the history points. These should always be given in three
    dimensions. \\
    \bottomrule
  \end{tabularx}
\end{center}

For example, to output the value of the solution fields at three points
$(1,0.5,0)$, $(2,0.5,0)$ and $(3,0.5,0)$ into a file \inltt{TimeValues.his}
every 10 timesteps, we use the syntax:

\begin{lstlisting}[style=XMLStyle,gobble=2]
  <FILTER TYPE="HistoryPoints">
      <PARAM NAME="OutputFile">TimeValues</PARAM>
      <PARAM NAME="OutputFrequency">10</PARAM>
      <PARAM NAME="Points">
          1 0.5 0
          2 0.5 0
          3 0.5 0
      </PARAM>
  </FILTER>
\end{lstlisting}

\subsection {ThresholdMax}

The threshold value filter writes a field output containing a variable $m$,
defined by the time at which the selected variable first exceeds a specified
threshold value. The default name of the output file is the name of the session
with the suffix \inlsh{\_max.fld}. Thresholding is applied based on the first
variable listed in the session by default.

The following parameters are supported:

\begin{center}
  \begin{tabularx}{0.99\textwidth}{lllX}
    \toprule
    \textbf{Option name} & \textbf{Required} & \textbf{Default} & 
    \textbf{Description} \\
    \midrule
    \inltt{OutputFile}      & \xmark   & \emph{session}\_max.fld &
    Output filename to which the threshold times are written.\\
    \inltt{ThresholdVar}    & \xmark   & \emph{first variable name} &
    Specifies the variable on which the threshold will be applied.\\
    \inltt{ThresholdValue}  & \cmark   & - &
    Specifies the threshold value.\\
    \inltt{InitialValue}    & \cmark   & - &
    Specifies the initial time.\\
    \inltt{StartTime}       & \xmark   & 0.0 &
    Specifies the time at which to start recording.\\
    \bottomrule
  \end{tabularx}
\end{center}
 
An example is given below:
 
\begin{lstlisting}[style=XMLStyle]
  <FILTER TYPE="ThresholdMax">
      <PARAM NAME="OutputFile"> threshold_max.fld </PARAM>
      <PARAM NAME="ThresholdVar"> u </PARAM>
      <PARAM NAME="ThresholdValue"> 0.1 </PARAM>
      <PARAM NAME="InitialValue">  0.4 </PARAM>
  </FILTER>
\end{lstlisting}

which produces a field file \inlsh{threshold\_max.fld}.

\subsection{ThresholdMin value}

Performs the same function as the \inltt{ThresholdMax} filter but records the
time at which the threshold variable drops below a prescribed value.

\subsection{One-dimensional energy}

This filter is designed to output the energy spectrum of one-dimensional
elements. It transforms the solution field at each timestep into a orthogonal
basis defined by the functions
\[
\psi_p(\xi) = L_p(\xi)
\]
where $L_p$ is the $p$-th Legendre polynomial. This can be used to show the
presence of, for example, oscillations in the underlying field due to numerical
instability. The resulting output is written into a file called
\inltt{session.eny} by default. The following parameters are supported:

\begin{center}
  \begin{tabularx}{0.99\textwidth}{lllX}
    \toprule
    \textbf{Option name} & \textbf{Required} & \textbf{Default} &
    \textbf{Description} \\
    \midrule
    \inltt{OutputFile}      & \xmark   & \inltt{session} &
    Prefix of the output filename to which the energy spectrum is written.\\
    \inltt{OutputFrequency} & \xmark   & 1 &
    Number of timesteps after which output is written.\\
    \bottomrule
  \end{tabularx}
\end{center}

An example syntax is given below:

\begin{lstlisting}[style=XMLStyle,gobble=2]
  <FILTER TYPE="Energy1D">
      <PARAM NAME="OutputFile">EnergyFile</PARAM>
      <PARAM NAME="OutputFrequency">10</PARAM>
  </FILTER>
\end{lstlisting}

\subsection{Modal energy}

\begin{notebox}
  This filter is only supported for the incompressible Navier-Stokes solver.
\end{notebox}

This filter calculates the time-evolution of the kinetic energy. In the case of
a two- or three-dimensional simulation this is defined as
\[
E_k(t) = \frac{1}{2} \int_{\Omega} \|\mathbf{u}\|^2\, dx
\]
However if the simulation is written as a one-dimensional homogeneous expansion
so that
\[
\mathbf{u}(\mathbf{x},t) = \sum_{k=0}^N \mathbf{\hat{u}}_k(t)e^{2\pi ik\mathbf{x}}
\]
then we instead calculate the energy spectrum
\[
E_k(t) = \frac{1}{2} \int_{\Omega} \|\mathbf{\hat{u}}_k\|^2\, dx.
\]
Note that in this case, each component of $\mathbf{\hat{u}}_k$ is a complex
number and therefore $N = \inltt{HomModesZ}/2$ lines are output for each
timestep. This is a particularly useful tool in examining turbulent and
transitional flows which use the homogeneous extension. In either case, the
resulting output is written into a file called \inltt{session.mdl} by default.

The following parameters are supported:

\begin{center}
  \begin{tabularx}{0.99\textwidth}{lllX}
    \toprule
    \textbf{Option name} & \textbf{Required} & \textbf{Default} & 
    \textbf{Description} \\
    \midrule
    \inltt{OutputFile}      & \xmark   & \inltt{session} &
    Prefix of the output filename to which the energy spectrum is written.\\
    \inltt{OutputFrequency} & \xmark   & 1 &
    Number of timesteps after which output is written.\\
    \bottomrule
  \end{tabularx}
\end{center}

An example syntax is given below:

\begin{lstlisting}[style=XMLStyle,gobble=2]
  <FILTER TYPE="ModalEnergy">
      <PARAM NAME="OutputFile">EnergyFile</PARAM>
      <PARAM NAME="OutputFrequency">10</PARAM>
  </FILTER>
\end{lstlisting}

\subsection{Aerodynamic forces}

\begin{notebox}
  This filter is only supported for the incompressible Navier-Stokes solver.
\end{notebox}

This filter evaluates the aerodynamic forces along a specific surface. The
forces are projected along the Cartesian axes and the pressure and viscous
contributions are computed in each direction.

The following parameters are supported:

\begin{center}
  \begin{tabularx}{0.99\textwidth}{lllX}
    \toprule
    \textbf{Option name} & \textbf{Required} & \textbf{Default} & 
    \textbf{Description} \\
    \midrule
    \inltt{OutputFile}      & \xmark   & \inltt{session} &
    Prefix of the output filename to which the forces are written.\\
    \inltt{Frequency}       & \xmark   & 1 &
    Number of timesteps after which output is written.\\
    \inltt{Boundary}        & \cmark   & - &
    Boundary surfaces on which the forces are to be evaluated.\\
    \bottomrule
  \end{tabularx}
\end{center}

An example is given below:

\begin{lstlisting}[style=XMLStyle]
  <FILTER TYPE="AeroForces">
      <PARAM NAME="OutputFile">DragLift.frc</PARAM>
      <PARAM NAME="OutputFrequency">10</PARAM>
      <PARAM NAME="Boundary"> B[1,2] </PARAM>		
  </FILTER>
\end{lstlisting}

During the execution a file named \inltt{DragLift.frc} will be created and the
value of the aerodynamic forces on boundaries 1 and 2, defined in the
\inltt{GEOMETRY} section, will be output every 10 time steps.

\subsection{Kinetic energy and enstrophy}

\begin{notebox}
  This filter is only supported for the incompressible and compressible
  Navier-Stokes solvers \textbf{in three dimensions}.
\end{notebox}

The purpose of this filter is to calculate the kinetic energy and enstrophy
%
\[
E_k = \frac{1}{2\mu(\Omega)}\int_{\Omega} \|\mathbf{u}\|^2\, dx, \qquad
\mathcal{E} = \frac{1}{2\mu(\Omega)}\int_{\Omega} \|\mathbf{\omega}\|^2\, dx
\]
%
where $\mu(\Omega)$ is the volume of the domain $\Omega$. This produces a file
containing the time-evolution of the kinetic energy and enstrophy fields. By
default this file is called \inltt{session.eny} where \inltt{session} is the
session name.

The following parameters are supported:
%
\begin{center}
  \begin{tabularx}{0.99\textwidth}{lllX}
    \toprule
    \textbf{Option name} & \textbf{Required} & \textbf{Default} & 
    \textbf{Description} \\
    \midrule
    \inltt{OutputFile}      & \xmark   & \texttt{session.eny} &
    Output file name to which the energy and enstrophy are written.\\
    \inltt{OutputFrequency} & \cmark   & - &
    Number of timesteps at which output is written.\\
    \bottomrule
  \end{tabularx}
\end{center}
%
To enable the filter, add the following to the \inltt{FILTERS} tag:
%
\begin{lstlisting}[style=XMLStyle,gobble=2]
  <FILTER TYPE="Energy">
      <PARAM NAME="OutputFrequency"> 1 </PARAM>
  </FILTER>
\end{lstlisting}


\section{Forcing}
An optional section of the file allows forcing functions to be defined. These are enclosed in the
\inltt{FORCING} tag. The forcing type is enclosed within the \inltt{FORCE} tag and expressed in the file as:

\begin{lstlisting}[style=XMLStyle] 
<FORCE TYPE="[NAME]">
    ...
</FORCE>
\end{lstlisting}

The force type can be any of the following:
\begin{itemize}
    \item "Absorption"
    \item "Body" 
    \item "Programmatic"
    \item "Noise"
\end{itemize}

\subsection{Absorption}
This force type allows the user to apply an absorption layer (essentially a porous region) anywhere in the domain. The user may also specify a velocity profile to be imposed at the start of this layer, and in the event of a time-dependent simulation, this profile can be modulated with a time-dependent function. These velocity functions and the function defining the region in which to apply the absorption layer are expressed in the \inltt{CONDITIONS} section, however the name of these functions are defined here by the \inltt{COEFF} tag for the layer, the \inltt{REFFLOW} tag for the velocity profile, and the \inltt{REFFLOWTIME} for the time-dependent function.  

\begin{lstlisting}[style=XMLStyle] 
<FORCE TYPE="Absorption">
    <COEFF> [FUNCTION NAME] <COEFF/>
    <REFFLOW> [FUNCTION NAME] <REFFLOW/>
    <REFFLOWTIME> [FUNCTION NAME] <REFFLOWTIME/>
</FORCE>
\end{lstlisting}


\subsection{Body}
This force type specifies the name of a body forcing function expressed in the \inltt{CONDITIONS} section.

\begin{lstlisting}[style=XMLStyle] 
<FORCE TYPE="Body">
    <BODYFORCE> [FUNCTION NAME] <BODYFORCE/>
</FORCE>
\end{lstlisting}

\subsection{Programmatic}
This force type allows a forcing function to be applied directly within the code, thus it has no associated function. 

\begin{lstlisting}[style=XMLStyle] 
<FORCE TYPE="Programmatic">
</FORCE>
\end{lstlisting}


\subsection{Noise}
This force type allows the user to specify the magnitude of a white noise force. 
Optional arguments can also be used to define the frequency in time steps to recompute the noise (default is never)
 and the number of time steps to apply the noise (default is the entire simulation).  

\begin{lstlisting}[style=XMLStyle] 
<FORCE TYPE="Noise">
    <WHITENOISE> [VALUE] <WHITENOISE/>
    <!-- Optional arguments -->
    <UPDATEFREQ> [VALUE] <UPDATEFREQ/>
    <NSTEPS> [VALUE] <NSTEPS/>
</FORCE>
\end{lstlisting}


\section{Analytic Expressions}
\label{sec:xml:analytic-expressions}

This section discusses particulars related to analytic expressions appearing in
Nektar++. Analytic expressions in Nektar++ are used to describe spatially or
temporally varying properties, for example
\begin{itemize}
\item velocity profiles on a boundary
\item some reference functions (e.g. exact solutions)
\end{itemize}
which can be retrieved in the solver code.

Analytic expressions appear as the content of \inltt{VALUE} attribute of
\begin{itemize}
\item boundary condition type tags within \inltt{<REGION>} subsection of
 \inltt{<BOUNDARYCONDITIONS>}, e.g. \inltt{<D>}, \inltt{<N>} etc. 
 %See [wiki:BoundaryConditionTypes] for details.
\item expression declaration tag \inltt{<E>} within \inltt{<FUNCTION>}
subsection.
\end{itemize}

The tags above declare analytic expressions as well as link them to one of the
field variables declared in \inltt{<EXPANSIONS>} section. For example, the
declaration 
\begin{lstlisting}[style=XMLStyle]
  <D VAR="u" VALUE="sin(PI*x)*cos(PI*y)" />
\end{lstlisting}
registers expression $\sin(\pi x)\cos(\pi y)$ as a Dirichlet
boundary constraint associated with field variable \inltt{u}.

Enforcing the same velocity profile at multiple boundary regions and/or field
variables results in repeated re-declarations of a corresponding analytic
expression. Currently one cannot directly link a boundary condition declaration
with an analytic expression uniquely specified somewhere else, e.g. in the
\inltt{<FUNCTION>} subsection. However this duplication does not affect an
overall computational performance.

% \subsection{Ordering of tags}
% 
% TODO Here one should describe the constraints of internal !SessionReader API for
% expression retrieval via function name and its ordering number. Ordering of tags
% is important for the user code. Not everything has a name for the solver code.

\subsection{Variables and coordinate systems}
Declarations of analytic expressions are formulated in terms of problem
space-time coordinates. The library code makes a number of assumptions to
variable names and their order of appearance in the declarations. This section
describes these assumptions.

Internally, the library uses 3D global coordinate space regardless of problem
dimension. Internal global coordinate system has natural basis
{{{(1,0,0),(0,1,0),(0,0,1)}}} with coordinates '''x''','''y''' and '''z'''. In
other words, variables '''x''','''y''' and '''z''' are considered to be first,
second and third coordinates of a point ('''x''','''y''','''z''').

Declarations of problem spatial variables do not exist in the current XML file
format. Even though field variables are declarable as in the following code
snippet, 
\begin{lstlisting}[style=XMLStyle]
   <VARIABLES>
     <V ID="0"> u </V>
     <V ID="1"> v </V>
   </VARIABLES>
\end{lstlisting} 
there are no analogous tags for space variables. However an attribute
\inlsh{SPACE} of \inlsh{<GEOMETRY>} section tag declares the dimension of
problem space. For example, \begin{lstlisting}[style=XMLStyle]
  <GEOMETRY DIM="1" SPACE="2"> ...
  </GEOMETRY>
\end{lstlisting}
specifies 1D flow within 2D problem space. The number of spatial variables
presented in expression declaration should match space dimension declared via
\inltt{<GEOMETRY>} section tag.

The library assumes the problem space also has natural basis and spatial
coordinates have names '''x''','''y''' and'''z'''.

Problem space is naturally embedded into the global coordinate space: each point
of
\begin{itemize}
\item 1D problem space with coordinate {{{x}}} is represented by 3D point
 {{{(x,0,0)}}} in the global coordinate system;
\item 2D problem space with coordinates {{{(x,y)}}} is represented by 3D point 
 {{{(x,y,0)}}} in the global coordinate system;
\item 3D problem space with coordinates {{{(x,y,z)}}} has the
 same coordinates in the global space coordinates.
\end{itemize}

Currently, there is no way to describe rotations and translations of problem
space relative to the global coordinate system.

The list of variables allowed in analytic expressions depends on the problem
dimension:
\begin{itemize}
\item For 1D problem analytic expressions must make use of variable '''x'''
only;
\item For 2D problem analytic expressions should make use of variables '''x'''
and '''y'''.
\item 3D problems may use variables '''x''', '''y''' and '''z''' in their
analytic expressions.
\end{itemize}

Violation of these constraints yields unpredictable results of expression
evaluation. The current implementation assigns magic value -9999 to each
dimensionally excessive spacial variable appearing in analytic expressions. For
example, the following declaration 
\begin{lstlisting}[style=XMLStyle]
  <GEOMETRY DIM="2" SPACE="2"> ...
  </GEOMETRY> ...
  <CONDITIONS> ...
    <BOUNDARYCONDITIONS>
       <REGION REF="0">
         <D VAR="u" VALUE="x+y+z" /> <D VAR="v" VALUE="sin(PI*x)*cos(PI*y)" />
       </REGION>
     </BOUNDARYCONDITIONS>
  ...
  </CONDITIONS>
\end{lstlisting}
results in expression $x+y+z$ being evaluated at spatial points
$(x_i,y_i, -9999)$ where $x_i$ and $y_i$ are
the spacial coordinates of boundary degrees of freedom. However, the library
behaviour under this constraint violation may change at later stages of
development (e.g., magic constant 0 may be chosen) and should be considered
unpredictable.

Another example of unpredictable behaviour corresponds to wrong ordering of
variables:
\begin{lstlisting}[style=XMLStyle]
  <GEOMETRY DIM="1" SPACE="1"> ...
  </GEOMETRY> ...
  <CONDITIONS> ...
    <BOUNDARYCONDITIONS>
       <REGION REF="0">
         <D VAR="u" VALUE="sin(y)" />
       </REGION>
     </BOUNDARYCONDITIONS>
  ...
  </CONDITIONS>
\end{lstlisting}
Here one declares 1D problem, so Nektar++ library assumes spacial variable
is '''x'''. At the same time, an expression $sin(y)$ is perfectly
valid on its own, but since it does not depend on '''x''', it will be evaluated
to constant $sin(-9999)$ regardless of degree of freedom under
consideration.

\subsubsection{Time dependence}

Variable '''t''' represents time dependence within analytic expressions. The
boundary condition declarations need to add an additional property
\inltt{USERDEFINEDTYPE="TimeDependent"} in order to flag time dependency to
the library.

% TODO:
%  * check there are no cases when the library evaluates analytic expressions with
%  non-zero time values even though {{{TimeDependent}}} property is not defined *
%  discuss time dependence of functions declared within {} section

\subsubsection{Syntax of analytic expressions}

Analytic expressions are formed of
\begin{itemize}
\item brackets {{{()}}}. Bracketing structure must be balanced.
\item real numbers: every representation is allowed that is correct for
\inlsh{boost::lexical\_cast<double>()}, e.g.
\begin{lstlisting}[style=XMLStyle]
   1.2, 1.2e-5, .02
\end{lstlisting}
\item mathematical constants
\begin{center}
\begin{tabular}{lcc}
\toprule
Identifier & Meaning & Real Value \\
\midrule
\multicolumn{3}{c}{\textbf{Fundamental constants}} \\
E           & Natural Logarithm     & 2.71828182845904523536 \\
PI          & $\pi$                 & 3.14159265358979323846 \\
GAMMA       & Euler Gamma           & 0.57721566490153286060 \\
DEG         & deg/radian            & 57.2957795130823208768 \\
PHI         & golden ratio          & 1.61803398874989484820 \\
\multicolumn{3}{c}{\textbf{Derived constants}} \\
LOG2E       & $\log_2 e$            & 1.44269504088896340740 \\
LOG10E      & $\log_{10} e$         & 0.43429448190325182765 \\
LN2         & $\log_e 2$            & 0.69314718055994530942 \\
PI\_2       & $\frac{\pi}{2}$       & 1.57079632679489661923 \\
PI\_4       & $\frac{\pi}{4}$       & 0.78539816339744830962 \\
1\_PI       & $\frac{1}{\pi}$       & 0.31830988618379067154 \\
2\_PI       & $\frac{2}{\pi}$       & 0.63661977236758134308 \\
2\_SQRTPI   & $\frac{2}{\sqrt{\pi}}$& 1.12837916709551257390 \\
SQRT2       & $\sqrt{2}$            & 1.41421356237309504880 \\
SQRT1\_2    & $\frac{1}{\sqrt{2}}$  & 0.70710678118654752440 \\
\bottomrule
\end{tabular}
\end{center}

\item parameters: alphanumeric names with underscores, e.g. \inltt{GAMMA\_123},
\inltt{GaM123\_45a\_}, \inltt{\_gamma123} are perfectly acceptable parameter
names. However parameter name cannot start with a numeral. Parameters must be
defined with \inltt{<PARAMETERS>...</PARAMETERS>}. Parameters play the role of
constants that may change their values in between of expression evaluations.

\item variables (i.e., \inlsh{x, y, z} and \inlsh{t})
\item unary minus operator (e.g. \inltt{-x})
\item binary arithmetic operators \inltt{+, -, *, /, \^{}}
   Powering operator allows using real exponents (it is implemented with
   \inlsh{std::pow()} function)
\item boolean comparison operations \inlsh{<, <=, >, >=, ==} evaluate their
sub-expressions to real values 0.0 or 1.0.
\item mathematical functions of one or two arguments:
\begin{center}
\begin{tabular}{ll}
  \toprule
  \textbf{Identifier} & \textbf{Meaning} \\
  \midrule
  \texttt{abs(x)}     & absolute value $|x|$ \\
  \texttt{asin(x)}    & inverse sine $\arcsin x$ \\
  \texttt{acos(x)}    & inverse cosine $\arccos x$ \\
  \texttt{ang(x,y)}   & computes polar coordinate $\theta=\arctan(y/x)$ from $(x,y)$\\
  \texttt{atan(x)}    & inverse tangent $\arctan x$ \\
  \texttt{atan2(y,x)} & inverse tangent function (used in polar transformations) \\
  \texttt{ceil(x)}    & round up to nearest integer $\lceil x\rceil$ \\
  \texttt{cos(x)}     & cosine $\cos x$ \\
  \texttt{cosh(x)}    & hyperbolic cosine $\cosh x$ \\
  \texttt{exp(x)}     & exponential $e^x$ \\
  \texttt{fabs(x)}    & absolute value (equivalent to \texttt{abs}) \\
  \texttt{floor(x)}   & rounding down $\lfloor x\rfloor$ \\
  \texttt{log(x)}     & logarithm base $e$, $\ln x = \log x$ \\
  \texttt{log10(x)}   & logarithm base 10, $\log_{10} x$ \\
  \texttt{rad(x,y)}   & computes polar coordinate $r=\sqrt{x^2+y^2}$ from $(x,y)$\\
  \texttt{sin(x)}     & sine $\sin x$ \\
  \texttt{sinh(x)}    & hyperbolic sine $\sinh x$ \\
  \texttt{sqrt(x)}    & square root $\sqrt{x}$ \\
  \texttt{tan(x)}     & tangent $\tan x$ \\
  \texttt{tanh(x)}    & hyperbolic tangent $\tanh x$ \\
  \bottomrule
\end{tabular}
\end{center}

These functions are implemented by means of the cmath library:
\url{http://www.cplusplus.com/reference/clibrary/cmath/}. Underlying data type
is \inltt{double} at each stage of expression evaluation. As consequence,
complex-valued expressions (e.g. $(-2)^0.123$) get value \inlsh{nan} (not a
number). The operator \inlsh{\^{}} is implemented via call to \inlsh{std::pow()}
function and accepts arbitrary real exponents.

\item random noise generation functions. Currently implemented is
\inltt{awgn(sigma)} - Gaussian Noise generator, where $\sigma$ is the variance
of normal distribution with zero mean. Implemented using the
\texttt{boost::mt19937} random number generator with boost variate generators
(see \url{http://www.boost.org/libs/random})
\end{itemize}


\subsubsection{Examples}
Some straightforward examples include
\begin{itemize}
\item Basic arithmetic operators: \inltt{0.5*0.3164/(3000\^{}0.25)}
\item Simple polynomial functions: \inltt{y*(1-y)}
\item Use of values defined in \inltt{PARAMETERS} section:
\inltt{-2*Kinvis*(x-1)}
\item More complex expressions involving trigonometric functions, parameters and
constants: \inltt{(LAMBDA/2/PI)*exp(LAMBDA*x)*sin(2*PI*y)}
\item Boolean operators for multi-domain functions:
\inltt{(y<0)*sin(y) + (y>=0)*y}
\end{itemize}

\subsection{Performance considerations}
Processing analytic expressions is split into two stages:
\begin{itemize}
\item parsing with pre-evaluation of constant sub-expressions,
\item evaluation to a number.
\end{itemize}
Parsing of analytic expressions with their partial evaluation take place at the
time of setting the run up (reading an XML file). Each analytic expression,
after being pre-processed, is stored internally and quickly retrieved when it
turns to evaluation at given spatial-time point(s). This allows to perform
evaluation of expressions at a large number of spacial points with minimal setup
costs.

\subsubsection{Pre-evaluation details}
Partial evaluation of all constant sub-expressions makes no sense in using
derived constants from table above. This means, either make use of pre-defined
constant \inlsh{LN10\^{}2} or straightforward expression \inlsh{log10(2)\^{}2}
results in constant \inlsh{5.3018981104783980105} being stored internally after
pre-processing. The rules of pre-evaluation are as follows:
\begin{itemize}
\item constants, numbers and their combinations with arithmetic, analytic and
 comparison operators are pre-evaluated,
\item appearance of a variable or parameter
 at any recursion level stops pre-evaluation of all upper level operations (but
 doesn't stop pre-evaluation of independent parallel sub-expressions).
\end{itemize}

For example, declaration 
\begin{lstlisting}[style=XMLStyle]
     <D VAR="u" VALUE="exp(-x*sin(PI*(sqrt(2)+sqrt(3))/2)*y )" />
\end{lstlisting}
results in expression \inlsh{exp(-x*(-0.97372300937516503167)*y )} being
stored internally: sub-expression \inlsh{sin(PI*(sqrt(2)+sqrt(3))/2)} is
evaluated to constant but appearance of \inlsh{x} and \inlsh{y} variables
stops further pre-evaluation.

Grouping predefined constants and numbers together helps. Its useful to put
brackets to be sure your constants do not run out and become factors of some
variables or parameters.

Expression evaluator does not do any clever simplifications of input
expressions, which is clear from example above (there is no point in double
negation). The following subsection addresses the simplification strategy.

\subsubsection{Preparing analytic expression}

The total evaluation cost depends on the overall number of operations. Since
evaluator is not making simplifications, it worth trying to minimise the total
number of operations in input expressions manually.

Some operations are more computationally expensive than others. In an order of
increasing complexity:
\begin{itemize}
\item \inlsh{+, -, <, >, <=, >=, ==, }
\item \inlsh{*, /, abs, fabs, ceil, floor,}
\item \inlsh{\^{}, sqrt, exp, log, log10, sin, cos, tan, sinh, cosh, tanh, asin,
acos, atan}.
\end{itemize}

For example,
\begin{itemize}
\item \inlsh{x*x} is faster than \inlsh{x\^{}2} --- it is one double
multiplication vs generic calculation of arbitrary power with real exponents.
\item \inlsh{(x+sin(y))\^{}2} is faster than \inlsh{(x+sin(y))*(x+sin(y))} -
sine is an expensive operation. It is cheaper to square complicated expression rather than
 compute it twice and add one multiplication.
\item An expression
\inltt{exp(-41*( (x+(0.3*cos(2*PI*t)))\^{}2 + (0.3*sin(2*PI*t))\^{}2 ))}
 makes use of 5 expensive operations (\inlsh{exp}, \inlsh{sin}, \inlsh{cos}
 and power \inlsh{\^{}} twice) while an equivalent expression
\inltt{exp(-41*( x*x+0.6*x*cos(2*PI*t) + 0.09 ))}
 uses only 2 expensive operations.
\end{itemize}

If any simplifying identity applies to input expression, it may worth applying
it, provided it minimises the complexity of evaluation. Computer algebra systems
may help.

\subsubsection{Vectorized evaluation}

Expression evaluator is able to calculate an expression for either given point
(its space-time coordinates) or given array of points (arrays of their
space-time coordinates, it uses SoA). Vectorized evaluation is faster then
sequential due to a better data access pattern. Some expressions give measurable
speedup factor $4.6$. Therefore, if you are creating your own solver, it
worth making vectorized calls.
%%% Local Variables: 
%%% mode: latex
%%% TeX-master: "../user-guide"
%%% End: 

