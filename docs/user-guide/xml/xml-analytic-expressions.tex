\section{Analytic Expressions}
This section discusses particulars related to analytic expressions appearing in
Nektar++. Analytic expressions in Nektar++ are used to describe spatially or
temporally varying properties, for example
\begin{itemize}
\item velocity profiles on a boundary
\item some reference functions (e.g. exact solutions)
\end{itemize}
which can be retrieved in the solver code.

Analytic expressions appear as the content of \inltt{VALUE} attribute of
\begin{itemize}
\item boundary condition type tags within \inltt{<REGION>} subsection of
 \inltt{<BOUNDARYCONDITIONS>}, e.g. \inltt{<D>}, \inltt{<N>} etc. 
 %See [wiki:BoundaryConditionTypes] for details.
\item expression declaration tag \inltt{<E>} within \inltt{<FUNCTION>}
subsection.
\end{itemize}

The tags above declare analytic expressions as well as link them to one of the
field variables declared in \inltt{<EXPANSIONS>} section. For example, the
declaration 
\begin{lstlisting}[style=XMLStyle]
  <D VAR="u" VALUE="sin(PI*x)*cos(PI*y)" />
\end{lstlisting}
registers expression $\sin(\pi x)\cos(\pi y)$ as a Dirichlet
boundary constraint associated with field variable \inltt{u}.

Enforcing the same velocity profile at multiple boundary regions and/or field
variables results in repeated re-declarations of a corresponding analytic
expression. Currently one cannot directly link a boundary condition declaration
with an analytic expression uniquely specified somewhere else, e.g. in the
\inltt{<FUNCTION>} subsection. However this duplication does not affect an
overall computational performance.

% \subsection{Ordering of tags}
% 
% TODO Here one should describe the constraints of internal !SessionReader API for
% expression retrieval via function name and its ordering number. Ordering of tags
% is important for the user code. Not everything has a name for the solver code.

\subsection{Variables and coordinate systems}
Declarations of analytic expressions are formulated in terms of problem
space-time coordinates. The library code makes a number of assumptions to
variable names and their order of appearance in the declarations. This section
describes these assumptions.

Internally, the library uses 3D global coordinate space regardless of problem
dimension. Internal global coordinate system has natural basis
{{{(1,0,0),(0,1,0),(0,0,1)}}} with coordinates '''x''','''y''' and '''z'''. In
other words, variables '''x''','''y''' and '''z''' are considered to be first,
second and third coordinates of a point ('''x''','''y''','''z''').

Declarations of problem spatial variables do not exist in the current XML file
format. Even though field variables are declarable as in the following code
snippet, 
\begin{lstlisting}[style=XMLStyle]
   <VARIABLES>
     <V ID="0"> u </V>
     <V ID="1"> v </V>
   </VARIABLES>
\end{lstlisting} 
there are no analogous tags for space variables. However an attribute
\inlsh{SPACE} of \inlsh{<GEOMETRY>} section tag declares the dimension of
problem space. For example, \begin{lstlisting}[style=XMLStyle]
  <GEOMETRY DIM="1" SPACE="2"> ...
  </GEOMETRY>
\end{lstlisting}
specifies 1D flow within 2D problem space. The number of spacial variables
presented in expression declaration should match space dimension declared via
\inltt{<GEOMETRY>} section tag.

The library assumes the problem space also has natural basis and spatial
coordinates have names '''x''','''y''' and'''z'''.

Problem space is naturally embedded into the global coordinate space: each point
of
\begin{itemize}
\item 1D problem space with coordinate {{{x}}} is represented by 3D point
 {{{(x,0,0)}}} in the global coordinate system;
\item 2D problem space with coordinates {{{(x,y)}}} is represented by 3D point 
 {{{(x,y,0)}}} in the global coordinate system;
\item 3D problem space with coordinates {{{(x,y,z)}}} has the
 same coordinates in the global space coordinates.
\end{itemize}

Currently, there is no way to describe rotations and translations of problem
space relative to the global coordinate system.

The list of variables allowed in analytic expressions depends on the problem
dimension:
\begin{itemize}
\item For 1D problem analytic expressions must make use of variable '''x'''
only;
\item For 2D problem analytic expressions should make use of variables '''x'''
and '''y'''.
\item 3D problems may use variables '''x''', '''y''' and '''z''' in their
analytic expressions.
\end{itemize}

Violation of these constraints yields unpredictable results of expression
evaluation. The current implementation assigns magic value -9999 to each
dimensionally excessive spacial variable appearing in analytic expressions. For
example, the following declaration 
\begin{lstlisting}[style=XMLStyle]
  <GEOMETRY DIM="2" SPACE="2"> ...
  </GEOMETRY> ...
  <CONDITIONS> ...
    <BOUNDARYCONDITIONS>
       <REGION REF="0">
         <D VAR="u" VALUE="x+y+z" /> <D VAR="v" VALUE="sin(PI*x)*cos(PI*y)" />
       </REGION>
     </BOUNDARYCONDITIONS>
  ...
  </CONDITIONS>
\end{lstlisting}
results in expression $x+y+z$ being evaluated at spatial points
$(x_i,y_i, -9999)$ where $x_i$ and $y_i$ are
the spacial coordinates of boundary degrees of freedom. However, the library
behaviour under this constraint violation may change at later stages of
development (e.g., magic constant 0 may be chosen) and should be considered
unpredictable.

Another example of unpredictable behaviour corresponds to wrong ordering of
variables:
\begin{lstlisting}[style=XMLStyle]
  <GEOMETRY DIM="1" SPACE="1"> ...
  </GEOMETRY> ...
  <CONDITIONS> ...
    <BOUNDARYCONDITIONS>
       <REGION REF="0">
         <D VAR="u" VALUE="sin(y)" />
       </REGION>
     </BOUNDARYCONDITIONS>
  ...
  </CONDITIONS>
\end{lstlisting}
Here one declares 1D problem, so Nektar++ library assumes spacial variable
is '''x'''. At the same time, an expression $sin(y)$ is perfectly
valid on its own, but since it does not depend on '''x''', it will be evaluated
to constant $sin(-9999)$ regardless of degree of freedom under
consideration.

\subsubsection{Time dependence}

Variable '''t''' represents time dependence within analytic expressions. The
boundary condition declarations need to add an additional property
\inltt{USERDEFINEDTYPE="TimeDependent"} in order to flag time dependency to
the library.

% TODO:
%  * check there are no cases when the library evaluates analytic expressions with
%  non-zero time values even though {{{TimeDependent}}} property is not defined *
%  discuss time dependence of functions declared within {} section

\subsubsection{Syntax of analytic expressions}

Analytic expressions are formed of
\begin{itemize}
\item brackets {{{()}}}. Bracketing structure must be balanced.
\item real numbers: every representation is allowed that is correct for
\inlsh{boost::lexical\_cast<double>()}, e.g.
\begin{lstlisting}[style=XMLStyle]
   1.2, 1.2e-5, .02
\end{lstlisting}
\item mathematical constants
\begin{center}
\begin{tabular}{lcc}
\toprule
Identifier & Meaning & Real Value \\
\midrule
\multicolumn{3}{c}{\textbf{Fundamental constants}} \\
E           & Natural Logarithm     & 2.71828182845904523536 \\
PI          & $\pi$                 & 3.14159265358979323846 \\
GAMMA       & Euler Gamma           & 0.57721566490153286060 \\
DEG         & deg/radian            & 57.2957795130823208768 \\
PHI         & golden ratio          & 1.61803398874989484820 \\
\multicolumn{3}{c}{\textbf{Derived constants}} \\
LOG2E       & $\log_2 e$            & 1.44269504088896340740 \\
LOG10E      & $\log_{10} e$         & 0.43429448190325182765 \\
LN2         & $\log_e 2$            & 0.69314718055994530942 \\
PI\_2       & $\frac{\pi}{2}$       & 1.57079632679489661923 \\
PI\_4       & $\frac{\pi}{4}$       & 0.78539816339744830962 \\
1\_PI       & $\frac{1}{\pi}$       & 0.31830988618379067154 \\
2\_PI       & $\frac{2}{\pi}$       & 0.63661977236758134308 \\
2\_SQRTPI   & $\frac{2}{\sqrt{\pi}}$& 1.12837916709551257390 \\
SQRT2       & $\sqrt{2}$            & 1.41421356237309504880 \\
SQRT1\_2    & $\frac{1}{\sqrt{2}}$  & 0.70710678118654752440 \\
\bottomrule
\end{tabular}
\end{center}

\item parameters: alphanumeric names with underscores, e.g. \inltt{GAMMA\_123,
GaM123\_45a\_, \_gamma123} are perfectly acceptable parameter names. However
parameter name cannot start with a numeral. Parameters must be defined with
\inltt{<PARAMETERS>...</PARAMETERS>}. Parameters play the role of constants
that may change their values in between of expression evaluations.

\item variables (i.e., \inlsh{x, y, z} and \inlsh{t})
\item unary minus operator (e.g. \inltt{-x}
\item binary arithmetic operators \inltt{+, -, *, /, \^}
   Powering operator allows using real exponents (it is implemented with
   \inlsh{std::pow()} function)
\item boolean comparison operations \inlsh{<, <=, >, >=, ==} evaluate their
sub-expressions to real values 0.0 or 1.0.
\item mathematical functions of one or two arguments:
\begin{center}
\begin{tabular}{ll}
  \toprule
  \textbf{Identifier} & \textbf{Meaning} \\
  \midrule
  \texttt{abs(x)}     & absolute value $|x|$ \\
  \texttt{asin(x)}    & inverse sine $\arcsin x$ \\
  \texttt{acos(x)}    & inverse cosine $\arccos x$ \\
  \texttt{ang(x,y)}   & computes polar coordinate $\theta=\arctan(y/x)$ from $(x,y)$\\
  \texttt{atan(x)}    & inverse tangent $\arctan x$ \\
  \texttt{atan2(y,x)} & inverse tangent function (used in polar transformations) \\
  \texttt{ceil(x)}    & round up to nearest integer $\lceil x\rceil$ \\
  \texttt{cos(x)}     & cosine $\cos x$ \\
  \texttt{cosh(x)}    & hyperbolic cosine $\cosh x$ \\
  \texttt{exp(x)}     & exponential $e^x$ \\
  \texttt{fabs(x)}    & absolute value (equivalent to \texttt{abs}) \\
  \texttt{floor(x)}   & rounding down $\lfloor x\rfloor$ \\
  \texttt{log(x)}     & logarithm base $e$, $\ln x = \log x$ \\
  \texttt{log10(x)}   & logarithm base 10, $\log_{10} x$ \\
  \texttt{rad(x,y)}   & computes polar coordinate $r=\sqrt{x^2+y^2}$ from $(x,y)$\\
  \texttt{sin(x)}     & sine $\sin x$ \\
  \texttt{sinh(x)}    & hyperbolic sine $\sinh x$ \\
  \texttt{sqrt(x)}    & square root $\sqrt{x}$ \\
  \texttt{tan(x)}     & tangent $\tan x$ \\
  \texttt{tanh(x)}    & hyperbolic tangent $\tanh x$ \\
  \bottomrule
\end{tabular}
\end{center}

These functions are implemented by means of the cmath library:
\url{http://www.cplusplus.com/reference/clibrary/cmath/}. Underlying data type
is \inltt{double} at each stage of expression evaluation. As consequence,
complex-valued expressions (e.g. $(-2)^0.123$) get value \inlsh{nan} (not a
number). The operator \inlsh{\^} is implemented via call to \inlsh{std::pow()}
function and accepts arbitrary real exponents.

\item random noise generation functions. Currently implemented is
\inltt{awgn(sigma)} - Gaussian Noise generator, where $\sigma$ is the variance
of normal distribution with zero mean. Implemented using the
\texttt{boost::mt19937} random number generator with boost variate generators
(see \url{http://www.boost.org/libs/random})
\end{itemize}

\subsection{Performance considerations}
Processing analytic expressions is split into two stages:
\begin{itemize}
\item parsing with pre-evaluation of constant sub-expressions,
\item evaluation to a number.
\end{itemize}
Parsing of analytic expressions with their partial evaluation take place at the
time of setting the run up (reading an XML file). Each analytic expression,
after being pre-processed, is stored internally and quickly retrieved when it
turns to evaluation at given spatial-time point(s). This allows to perform
evaluation of expressions at a large number of spacial points with minimal setup
costs.

\subsubsection{Pre-evaluation details}
Partial evaluation of all constant sub-expressions makes no sense in using
derived constants from table above. This means, either make use of pre-defined
constant \inlsh{LN10\^2} or straightforward expression \inlsh{log10(2)\^2}
results in constant \inlsh{5.3018981104783980105} being stored internally
after pre-processing. The rules of pre-evaluation are as follows:
\begin{itemize}
\item constants, numbers and their combinations with arithmetic, analytic and
 comparison operators are pre-evaluated,
\item appearance of a variable or parameter
 at any recursion level stops pre-evaluation of all upper level operations (but
 doesn't stop pre-evaluation of independent parallel sub-expressions).
\end{itemize}

For example, declaration 
\begin{lstlisting}[style=XMLStyle]
     <D VAR="u" VALUE="exp(-x*sin(PI*(sqrt(2)+sqrt(3))/2)*y )" />
\end{lstlisting}
results in expression \inlsh{exp(-x*(-0.97372300937516503167)*y )} being
stored internally: sub-expression \inlsh{sin(PI*(sqrt(2)+sqrt(3))/2)} is
evaluated to constant but appearance of \inlsh{x} and \inlsh{y} variables
stops further pre-evaluation.

Grouping predefined constants and numbers together helps. Its useful to put
brackets to be sure your constants do not run out and become factors of some
variables or parameters.

Expression evaluator does not do any clever simplifications of input
expressions, which is clear from example above (there is no point in double
negation). The following subsection addresses the simplification strategy.

\subsubsection{Preparing analytic expression}

The total evaluation cost depends on the overall number of operations. Since
evaluator is not making simplifications, it worth trying to minimise the total
number of operations in input expressions manually.

Some operations are more computationally expensive than others. In an order of
increasing complexity:
\begin{itemize}
\item \inlsh{+, -, <, >, <=, >=, ==, }
\item \inlsh{*, /, abs, fabs, ceil, floor,}
\item \inlsh{\^, sqrt, exp, log, log10, sin, cos, tan, sinh, cosh, tanh, asin,
acos, atan}.
\end{itemize}

For example,
\begin{itemize}
\item \inlsh{x*x} is faster than \inlsh{x\^2} --- it is one double
multiplication vs generic calculation of arbitrary power with real exponents.
\item \inlsh{(x+sin(y))\^2} is faster than \inlsh{(x+sin(y))*(x+sin(y))} -
sine is an expensive operation. It is cheaper to square complicated expression rather than
 compute it twice and add one multiplication.
\item An expression
\inltt{exp(-41*( (x+(0.3*cos(2*PI*t)))\^2 + (0.3*sin(2*PI*t))\^2 ))}
 makes use of 5 expensive operations (\inlsh{exp}, \inlsh{sin}, \inlsh{cos}
 and power \inlsh{\^} twice) while an equivalent expression
\inltt{exp(-41*( x*x+0.6*x*cos(2*PI*t) + 0.09 ))}
 uses only 2 expensive operations.
\end{itemize}

If any simplifying identity applies to input expression, it may worth applying
it, provided it minimises the complexity of evaluation. Computer algebra systems
may help.

\subsubsection{Vectorized evaluation}

Expression evaluator is able to calculate an expression for either given point
(its space-time coordinates) or given array of points (arrays of their
space-time coordinates, it uses SoA). Vectorized evaluation is faster then
sequential due to a better data access pattern. Some expressions give measurable
speedup factor $4.6$. Therefore, if you are creating your own solver, it
worth making vectorized calls.
%%% Local Variables: 
%%% mode: latex
%%% TeX-master: "../user-guide"
%%% End: 
