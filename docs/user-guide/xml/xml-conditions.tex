\section{Conditions}
The final section of the file defines parameters and boundary conditions which
define the nature of the problem to be solved. These are enclosed in the
\inltt{CONDITIONS} tag.

\subsection{Parameters}

Parameters may be required by a particular solver (for instance time-integration
parameters or solver-specific parameters), or arbitrary and only used within the
context of the session file (e.g. parameters in the definition of an initial
condition). All parameters are enclosed in the \inltt{PARAMETERS} XML element.

\begin{lstlisting}[style=XMLStyle] 
<PARAMETERS>
    ...
</PARAMETERS>
\end{lstlisting}

A parameter may be of integer or real type and may reference other parameters
defined previous to it. It is expressed in the file as

\begin{lstlisting}[style=XMLStyle]
<P> [PARAMETER NAME] = [PARAMETER VALUE] </P>
\end{lstlisting}

For example,

\begin{lstlisting}[style=XMLStyle]
<P> NumSteps = 1000              </P>
<P> TimeStep = 0.01              </P>
<P> FinTime  = NumSteps*TimeStep </P>
\end{lstlisting}

\subsection{Solver Information}

These specify properties to define the actions specific to solvers, typically
including the equation to solve, the projection type and the method of time
integration. The property/value pairs are specified as XML attributes. For
example, 
\begin{lstlisting}[style=XMLStyle] 
<SOLVERINFO>
  <I PROPERTY="EQTYPE"                VALUE="UnsteadyAdvection"    /> 
  <I PROPERTY="Projection"            VALUE="Continuous"           /> 
  <I PROPERTY="TimeIntegrationMethod" VALUE="ClassicalRungeKutta4" />
</SOLVERINFO>
\end{lstlisting}

The list of available solvers in Nektar++ can be found in
Part~\ref{p:applications}.

\subsubsection{Drivers}
Drivers are defined under the \inltt{CONDITIONS} section as properties of the 
\inltt{SOLVERINFO} XML element. The role of a driver is to manage the solver 
execution from an upper level. 

The default driver is called \inltt{Standard} and executes the following steps:
\begin{enumerate}
\item Prints out on screen a summary of all the conditions defined in the input file.
\item Sets up the initial and boundary conditions.
\item Calls the solver defined by \inltt{SolverType}  in the \inltt{SOLVERINFO} XML element.
\item Writes the results in the output (.fld) file.
\end{enumerate}

In the following example, the driver \inltt{Standard} is used to manage the 
execution of the incompressible Navier-Stokes equations:
\begin{lstlisting}[style=XMLStyle]
<SOLVERINFO>
    <I PROPERTY="EQTYPE" VALUE="UnsteadyNavierStokes" />
    <I PROPERTY="SolverType" VALUE="VelocityCorrectionScheme" />
    <I PROPERTY="Projection" VALUE="Galerkin" />
    <I PROPERTY="TimeIntegrationMethod" VALUE="IMEXOrder2" />
    <I PROPERTY="Driver" VALUE="Standard" />
</SOLVERINFO>
\end{lstlisting}

If no driver is specified in the session file, the driver \inltt{Standard} is 
called by default. Other drivers can be used and are typically focused on
specific applications. As described in Sec.
\ref{SectionIncNS_SolverInfo} and  \ref{SectionIncNS_SolverInfo_Stab}, 
the other possibilities are:
\begin{itemize}
\item \inltt{ModifiedArnoldi}  - computes of the leading eigenvalues and 
eigenmodes using modified Arnoldi method.
\item \inltt{Arpack} - computes of eigenvalues/eigenmodes using Implicitly 
Restarted Arnoldi Method (ARPACK).
\item \inltt{SteadyState} - uses the Selective Frequency Damping method 
(see Sec. \ref{SectionSFD}) to obtain a steady-state solution of the 
Navier-Stokes equations (compressible or incompressible).
\end{itemize}


\subsection{Variables}

These define the number (and name) of solution variables. Each variable is
prescribed a unique ID. For example a two-dimensional flow simulation may define
the velocity variables using

\begin{lstlisting}[style=XMLStyle]
<VARIABLES>
  <V ID="0"> u </V>
  <V ID="1"> v </V>
</VARIABLES>
\end{lstlisting}

\subsection{Global System Solution Information}

Many \nekpp solvers use an implicit formulation of their equations to, for
instance, improve timestep restrictions. This means that a large matrix system
must be constructed and a global system set up to solve for the unknown
coefficients. There are several approaches in the spectral/$hp$ element method
that can be used in order to improve efficiency in these methods, as well as
considerations as to whether the simulation is run in parallel or serial. \nekpp
opts for `sensible' default choices, but these may or may not be optimal
depending on the problem under consideration.

This section of the XML file therefore allows the user to specify the global
system solution parameters, which dictates the type of solver to be used for any
implicit systems that are constructed. This section is particularly useful when
using a multi-variable solver such as the incompressible Navier-Stokes solver,
as it allows us to select different preconditioning and residual convergence
options for each variable. As an example, consider the block defined by:

\begin{lstlisting}[style=XMLStyle]
<GLOBALSYSSOLNINFO>
  <V VAR="u,v,w">
    <I PROPERTY="GlobalSysSoln"             VALUE="IterativeStaticCond" />
    <I PROPERTY="Preconditioner"            VALUE="LowEnergyBlock"/>
    <I PROPERTY="IterativeSolverTolerance"  VALUE="1e-8"/>
  </V>
  <V VAR="p">
    <I PROPERTY="GlobalSysSoln"             VALUE="IterativeStaticCond" />
    <I PROPERTY="Preconditioner"     VALUE="FullLinearSpaceWithLowEnergyBlock"/>
    <I PROPERTY="IterativeSolverTolerance"  VALUE="1e-6"/>
  </V>
</GLOBALSYSSOLNINFO>
\end{lstlisting}

The above section specifies that the variables \texttt{u,v,w} should use the
\texttt{IterativeStaticCond} global solver alongside the \texttt{LowEnergyBlock}
preconditioner and an iterative tolerance of $10^{-8}$ on the residuals. However
the pressure variable \texttt{p} is generally stiffer: we therefore opt for a
more expensive \texttt{FullLinearSpaceWithLowEnergyBlock} preconditioner and a
larger residual of $10^{-6}$. We now outline the choices that one can use for
each of these parameters and give a brief description of what they mean.

\begin{notebox}
  Defaults for all fields can be defined by setting the equivalent property in
  the \texttt{SOLVERINFO} section. Parameters defined in this section will
  override any options specificed there.
\end{notebox}

\subsubsection{\texttt{GlobalSysSoln} options}

\nekpp presently implements four methods of solving a global system:

\begin{itemize}
  \item \textbf{Direct} solvers construct the full global matrix and directly
  invert it using an appropriate matrix technique, such as Cholesky
  factorisation, depending on the properties of the matrix. Direct solvers
  \textbf{only} run in serial.
  \item \textbf{Iterative} solvers instead apply matrix-vector multiplications
  repeatedly, using the conjugate gradient method, to converge to a solution to
  the system. For smaller problems, this is typically slower than a direct
  solve. However, for larger problems it can be used to solve the system in
  parallel execution.
  \item \textbf{Xxt} solvers use the $XX^T$ library to perform a parallel direct
  solve. This option is only available if the \texttt{NEKTAR\_USE\_MPI} option
  is enabled in the CMake configuration.
  \item \textbf{PETSc} solvers use the PETSc library, giving access to a wide
  range of solvers and preconditioners. See section~\ref{sec:petsc} below for
  some additional information on how to use the PETSc solvers. This option is
  only available if the \texttt{NEKTAR\_USE\_PETSC} option is enabled in the
  CMake configuration.
\end{itemize}

Both the \textbf{Xxt} and \textbf{PETSc} solvers are considered advanced and are
under development -- either the direct or iterative solvers are recommended in
most scenarios.

These solvers can be run in one of three approaches:

\begin{itemize}
  \item The \textbf{Full} approach constructs the global system based on all of
  the degrees of freedom contained within an element. For most of the \nekpp
  solvers, this technique is not recommended.
  \item The \textbf{StaticCond} approach applies a technique called \emph{static
    condensation} to instead construct the system using only the degrees of
  freedom on the boundaries of the elements, which reduces the system size
  considerably. This is the \textbf{default option in parallel}.
  \item \textbf{MultiLevelStaticCond} methods apply the static condensation
  technique repeatedly to further reduce the system size, which can improve
  performance by 25-30\% over the normal static condensation method. It is
  therefore the \textbf{default option in serial}. Note that whilst parallel
  execution technically works, this is under development and is likely to be
  slower than single-level static condensation: this is therefore not
  recommended.
\end{itemize}

The \texttt{GlobalSysSoln} option is formed by combining the method of solution
with the approach: for example \texttt{IterativeStaticCond} or
\texttt{PETScMultiLevelStaticCond}.

\subsubsection{Preconditioner options}

Preconditioners can be used in the iterative and PETSc solvers to reduce the
number of iterations needed to converge to the solution. There are a number of
preconditioner choices, the default being a simple Jacobi (or diagonal)
preconditioner, which is enabled by default. There are a number of choices that
can be enabled through this parameter, which are all generally discretisation
and dimension-dependent:

\begin{center}
  \begin{tabular}{lll}
    \toprule
    \textbf{Name}  & \textbf{Dimensions} & \textbf{Discretisations} \\
    \midrule
    \inltt{Null}                              & All  & All \\
    \inltt{Diagonal}                          & All  & All \\
    \inltt{FullLinearSpace}                   & 2/3D & CG  \\
    \inltt{LowEnergyBlock}                    & 3D   & CG  \\
    \inltt{Block}                             & 2/3D & All \\
    \midrule
    \inltt{FullLinearSpaceWithDiagonal}       & All  & CG  \\
    \inltt{FullLinearSpaceWithLowEnergyBlock} & 2/3D & CG  \\
    \inltt{FullLinearSpaceWithBlock}          & 2/3D & CG  \\
    \bottomrule
  \end{tabular}
\end{center}

For a detailed discussion of the mathematical formulation of these options, see
the developer guide.

\subsubsection{SuccessiveRHS options}

The \texttt{SuccessiveRHS} option can be used in the iterative solver only, to
attempt to reduce the number of iterations taken to converge to a solution. It
stores a number of previous solutions, dictated by the setting of the
\texttt{SuccessiveRHS} option, to give a better initial guess for the iterative
process.

\subsubsection{PETSc options and configuration}
\label{sec:petsc}

The PETSc solvers, although currently experimental, are operational both in
serial and parallel. PETSc gives access to a wide range of alternative solver
options such as GMRES, as well as any packages that PETSc can link against, such
as the direct multi-frontal solver MUMPS.

Configuration of PETSc options using its command-line interface dictates what
matrix storage, solver type and preconditioner should be used. This should be
specified in a \texttt{.petscrc} file inside your working directory, as command
line options are not currently passed through to PETSc to avoid conflict with
\nekpp options. As an example, to select a GMRES solver using an algebraic
multigrid preconditioner, and view the residual convergence, one can use the
configuration:

\begin{lstlisting}[style=BashInputStyle]
-ksp_monitor
-ksp_view
-ksp_type gmres
-pc_type gamg
\end{lstlisting}

Or to use MUMPS, one could use the options:

\begin{lstlisting}[style=BashInputStyle]
-ksp_type preonly
-pc_type lu
-pc_factor_mat_solver_package mumps
-mat_mumps_icntl_7 2
\end{lstlisting}

A final choice that can be specified is whether to use a \emph{shell}
approach. By default, \nekpp will construct a PETSc sparse matrix (or whatever
matrix is specified on the command line). This may, however, prove suboptimal
for higher order discretisations. In this case, you may choose to use the \nekpp
matrix-vector operators, which by default use an assembly approach that can
prove faster, by setting the \texttt{PETScMatMult} \texttt{SOLVERINFO} option to
\texttt{Shell}:

\begin{lstlisting}[style=XMLStyle]
<I PROPERTY="PETScMatMult" VALUE="Shell" />
\end{lstlisting}

The downside to this approach is that you are now constrained to using one of
the \nekpp preconditioners. However, this does give access to a wider range of
Krylov methods than are available inside \nekpp for more advanced users.

\subsection{Boundary Regions and Conditions}

Boundary conditions are defined by two XML elements. The first defines the
boundary regions in the domain in terms of composite entities from the
\inltt{GEOMETRY} section of the file. Each boundary region has a unique ID and
are defined as, 
\begin{lstlisting}[style=XMLStyle]
<BOUNDARYREGIONS>
    <B ID=[id]> [composite-list] </B>
    ...
</BOUNDARYREGIONS>
\end{lstlisting}
For example,
\begin{lstlisting}[style=XMLStyle]
<BOUNDARYREGIONS>
  <B ID="0"> C[2] </B>
  <B ID="1"> C[3] </B>
</BOUNDARYREGIONS>
\end{lstlisting}

The second XML element defines, for each variable, the condition to impose on
each boundary region, and has the form,
\begin{lstlisting}[style=XMLStyle]
<BOUNDARYCONDITIONS>
    <REGION REF="[regionID]">
      <[type1] VAR="[variable1]" VALUE="[expression1]" />
      ...
      <[typeN] VAR="[variableN]" VALUE="[expressionN]" />
    </REGION>
    ...
</BOUNDARYCONDITIONS>
\end{lstlisting}
There should be precisely one \inltt{REGION} entry for each \inltt{B} entry
defined in the \inltt{BOUNDARYREGION} section above. For example, to impose a
Dirichlet condition on both variables for a domain with a single region, 
\begin{lstlisting}[style=XMLStyle] 
<BOUNDARYCONDITIONS>
  <REGION REF="0">
    <D VAR="u" VALUE="sin(PI*x)*cos(PI*y)" /> 
    <D VAR="v" VALUE="sin(PI*x)*cos(PI*y)" />
  </REGION>
</BOUNDARYCONDITIONS>
\end{lstlisting}
Boundary condition specifications may refer to any parameters defined in the
session file. The \inltt{REF} attribute corresponds to a defined boundary
region. The tag used for each variable specifies the type of boundary condition
to enforce.

\subsubsection{Dirichlet (essential) condition}
Dirichlet conditions are specified with the \inltt{D} tag.

\begin{tabular}{llll}
Projection & Homogeneous support & Time-dependent support & Dimensions \\
\toprule
CG & Yes & Yes & 1D, 2D and 3D \\
DG & Yes & Yes & 1D, 2D and 3D \\
HDG& Yes & Yes & 1D, 2D and 3D
\end{tabular}

Example:
\begin{lstlisting}[style=XMLStyle]
<!-- homogeneous condition -->
<D VAR="u" VALUE="0" />
<!-- inhomogeneous condition -->
<D VAR="u" VALUE="x^2+y^2+z^2" />
<!-- time-dependent condition -->
<D VAR="u" USERDEFINEDTYPE="TimeDependent" VALUE="x+t" />
\end{lstlisting}

\subsubsection{Neumann (natural) condition}
Neumann conditions are specified with the \inltt{N} tag.

\begin{tabular}{llll}
Projection & Homogeneous support & Time-dependent support & Dimensions \\
\toprule
CG & Yes & Yes & 1D, 2D and 3D \\
DG & No  & No  & 1D, 2D and 3D \\
HDG & ? & ? & ?
\end{tabular}

Example:
\begin{lstlisting}[style=XMLStyle]
<!-- homogeneous condition -->
<N VAR="u" VALUE="0" />
<!-- inhomogeneous condition -->
<N VAR="u" VALUE="x^2+y^2+z^2" />
<!-- time-dependent condition -->
<N VAR="u" USERDEFINEDTYPE="TimeDependent" VALUE="x+t" />
<!-- high-order pressure boundary condition (for IncNavierStokesSolver) -->
<N VAR="u" USERDEFINEDTYPE="H" VALUE="0" />
\end{lstlisting}

\subsubsection{Periodic condition}
Periodic conditions are specified with the \inltt{P} tag.

\begin{tabular}{lll}
Projection & Homogeneous support & Dimensions \\
\toprule
CG  & Yes & 1D, 2D and 3D \\
DG  & No  & 2D and 3D
\end{tabular}

Example:
\begin{lstlisting}[style=XMLStyle]
<BOUNDARYREGIONS>
  <B ID="0"> C[1] </B>
  <B ID="1"> C[2] </B>
</BOUNDARYREGIONS>

<BOUNDARYCONDITIONS>
  <REGION REF="0">
    <P VAR="u" VALUE="[1]" />
  </REGION>
  <REGION REF="1">
    <P VAR="u" VALUE="[0]" />
  </REGION>
</BOUNDARYCONDITIONS>
\end{lstlisting}

Periodic boundary conditions are specified in a significantly different form to
other conditions. The \inltt{VALUE} property is used to specify which
\inltt{BOUNDARYREGION} is periodic with the current region in square brackets.

Caveats:
\begin{itemize}
\item A periodic condition must be set for '''both''' boundary regions; simply
 specifying a condition for region 0 or 1 in the above example is not enough.
\item The order of the elements inside the composites defining periodic
boundaries is important. For example, if `C[0]` above is defined as edge IDs 
`{0,5,4,3}` and `C[1]` as `{7,12,2,1}` then edge 0 is periodic with edge 7, 5 
with 12, and so on.
\item For the above reason, the composites must also therefore be of the same
size.
\item In three dimensions, care must be taken to correctly align triangular
faces which are intended to be periodic. The top (degenerate) vertex should be 
aligned so that, if the faces were connected, it would lie at the same point on 
both triangles.
\end{itemize}

\subsubsection{Time-dependent boundary conditions}
Time-dependent boundary conditions may be specified through setting the
\inltt{USERDEFINEDTYPE} attribute and using the parameter \inltt{t} where the
current time is required. For example,
\begin{lstlisting}[style=XMLStyle]
<D VAR="u" USERDEFINEDTYPE="TimeDependent" VALUE="sin(PI*(x-t))" />
\end{lstlisting}

\subsubsection{Boundary conditions from file}
Boundary conditions can also be loaded from file. The following example is from
the Incompressible Navier-Stokes solver,
\begin{lstlisting}[style=XMLStyle]
<REGION REF="1">
  <D VAR="u" FILE="Test_ChanFlow2D_bcsfromfiles_u_1.bc" />
  <D VAR="v" VALUE="0" />
  <N VAR="p" USERDEFINEDTYPE="H" VALUE="0" />
</REGION>
\end{lstlisting}

Boundary conditions can also be loaded simultaneously from a file and from an 
analytic expression. For example in the scenario where a spatial boundary 
condition is read from a file, but needs to be modulated by a time-dependent 
analytic expression:
\begin{lstlisting}[style=XMLStyle]
<REGION REF="1">
  <D VAR="u" VALUE="USERDEFINEDTYPE="TimeDependent" VALUE="sin(PI*(x-t))"
             FILE="bcsfromfiles_u_1.bc" />
</REGION>
\end{lstlisting}

In the case where both \inltt{VALUE} and \inltt{FILE} are specified, the values
are multiplied together to give the final value for the boundary condition. 

\subsection{Functions}

Finally, multi-variable functions such as initial conditions and analytic
solutions may be specified for use in, or comparison with, simulations. These
may be specified using expressions (\inltt{<E>}) or imported from a file
(\inltt{<F>}) using the Nektar++ FLD file format

\begin{lstlisting}[style=XMLStyle]
<FUNCTION NAME="ExactSolution">
  <E VAR="u" VALUE="sin(PI*x-advx*t))*cos(PI*(y-advy*t))" />
</FUNCTION>
<FUNCTION NAME="InitialConditions">
  <F VAR="u" FILE="session.rst" />
</FUNCTION>
\end{lstlisting}

A restart file is a solution file (in other words an .fld renamed as .rst) where
the field data is specified. The expansion order used to generate the .rst file
must be the same as that for the simulation.
.pts files contain scattered point data which needs to be interpolated to the field.
For further information on the file format and the different interpolation schemes, see
section~\ref{s:utilities:fieldconvert:sub:interppointdatatofld}.
All filenames must be specified relative to the location of the .xml file.

With the additional argument \inltt{TIMEDEPENDENT="1"}, different files can be
loaded for each timestep. The filenames are defined using
\href{http://www.boost.org/doc/libs/1_56_0/libs/format/doc/format.html#syntax}{boost::format syntax}
where the step time is used as variable. For example, the function
\inltt{Baseflow} would load the files \inltt{U0V0\_1.00000000E-05.fld},
\inltt{U0V0\_2.00000000E-05.fld} and so on.

\begin{lstlisting}[style=XMLStyle]
<FUNCTION NAME="Baseflow">
       <F VAR="U0,V0" TIMEDEPENDENT="1"FILE="U0V0_%14.8E.fld"/>
</FUNCTION>
\end{lstlisting}

For .pts files, the time consuming computation of interpolation weights is only
performed for the first timestep. The weights are stored and reused in all subsequent steps, 
which is why all consecutive .pts files must use the same ordering, number and location of
data points.

Other examples of this input feature can be the insertion of a forcing term,

\begin{lstlisting}[style=XMLStyle]
<FUNCTION NAME="BodyForce">
  <E VAR="u" VALUE="0" />
  <E VAR="v" VALUE="0" />
</FUNCTION>
<FUNCTION NAME="Forcing">
  <E VAR="u" VALUE="-(Lambda + 2*PI*PI)*sin(PI*x)*sin(PI*y)" />
</FUNCTION>
\end{lstlisting}

or of a linear advection term

\begin{lstlisting}[style=XMLStyle]
<FUNCTION NAME="AdvectionVelocity">
  <E VAR="Vx" VALUE="1.0" />
  <E VAR="Vy" VALUE="1.0" />
  <E VAR="Vz" VALUE="1.0" />
</FUNCTION>
\end{lstlisting}

\subsubsection{Remapping variable names}

Note that it is sometimes the case that the variables being used in the solver
do not match those saved in the FLD file. For example, if one runs a
three-dimensional incompressible Navier-Stokes simulation, this produces an FLD
file with the variables \inltt{u}, \inltt{v}, \inltt{w} and \inltt{p}. If we
wanted to use this velocity field as input for an advection velocity, the
advection-diffusion-reaction solver expects the variables \inltt{Vx}, \inltt{Vy}
and \inltt{Vz}.

We can manually specify this mapping by adding a colon to the

\begin{lstlisting}[style=XMLStyle]
<FUNCTION NAME="AdvectionVelocity">
  <F VAR="Vx,Vy,Vz" FILE="file.fld:u,v,w" />
</FUNCTION>
\end{lstlisting}

There are some caveats with this syntax:

\begin{itemize}
  \item You must specify the same number of fields for both the variable, and
  after the colon. For example, the following is not valid.
  \begin{lstlisting}[style=XMLStyle,gobble=4]
    <FUNCTION NAME="AdvectionVelocity">
      <F VAR="Vx,Vy,Vz" FILE="file.fld:u" />
    </FUNCTION>\end{lstlisting}
  \item This syntax is not valid with the wildcard operator \inltt{*}, so one
  cannot write for example:
  \begin{lstlisting}[style=XMLStyle,gobble=4]
    <FUNCTION NAME="AdvectionVelocity">
      <F VAR="*" FILE="file.fld:u,v,w" />
    </FUNCTION>
  \end{lstlisting}
\end{itemize}

\subsubsection{Time-dependent file-based functions}

With the additional argument \inltt{TIMEDEPENDENT="1"}, different files can be
loaded for each timestep. The filenames are defined using
\href{http://www.boost.org/doc/libs/1_56_0/libs/format/doc/format.html#syntax}{boost::format
  syntax} where the step time is used as variable. For example, the function
\inltt{Baseflow} would load the files \inltt{U0V0\_1.00000000E-05.fld},
\inltt{U0V0\_2.00000000E-05.fld} and so on.

\begin{lstlisting}[style=XMLStyle]
<FUNCTION NAME="Baseflow">
  <F VAR="U0,V0" TIMEDEPENDENT="1" FILE="U0V0_%14.8R.fld" />
</FUNCTION>
\end{lstlisting}

Section~\ref{sec:xml:analytic-expressions} provides the list of acceptable
mathematical functions and other related technical details.

\subsection{Quasi-3D approach}

To generate a Quasi-3D appraoch with Nektar++ we only need to create a 2D or a
1D mesh, as reported above, and then specify the parameters to extend the
problem to a 3D case. For a 2D spectral/hp element problem, we have a 2D mesh
and along with the parameters we need to define the problem (i.e. equation type,
boundary conditions, etc.). The only thing we need to do, to extend it to a
Quasi-3D approach, is to specify some additional parameters which characterise
the harmonic expansion in the third direction. First we need to specify in the
solver information section that that the problem will be extended to have one
homogeneouns dimension; here an example

\begin{lstlisting}[style=XMLStyle]
<SOLVERINFO>
  <I PROPERTY="SolverType"            VALUE="VelocityCorrectionScheme"/>
  <I PROPERTY="EQTYPE"                VALUE="UnsteadyNavierStokes"/>
  <I PROPERTY="AdvectionForm"         VALUE="Convective"/>
  <I PROPERTY="Projection"            VALUE="Galerkin"/>
  <I PROPERTY="TimeIntegrationMethod" VALUE="IMEXOrder2"/>
  <I PROPERTY="HOMOGENEOUS"           VALUE="1D"/>
</SOLVERINFO>
\end{lstlisting}

then we need to specify the parameters which define the 1D harmonic expanson
along the z-axis, namely the homogeneous length (LZ) and the number of modes in
the homogeneous direction (HomModesZ). HomModesZ corresponds also to the number
of quadrature points in the homogenous direction, hence on the number of 2D
planes discretized with a specral/hp element method.

\begin{lstlisting}[style=XMLStyle]
<PARAMETERS>
  <P> TimeStep      = 0.001   </P>
  <P> NumSteps      = 1000    </P>
  <P> IO_CheckSteps = 100     </P>
  <P> IO_InfoSteps  = 10      </P>
  <P> Kinvis        = 0.025   </P>
  <P> HomModesZ     = 4       </P>
  <P> LZ            = 1.0     </P>
</PARAMETERS>
\end{lstlisting}

In case we want to create a Quasi-3D approach starting form a 1D spectral/hp
element mesh, the procedure is the same, but we need to specify the parameters
for two harmonic directions (in Y and Z direction). For Example,

\begin{lstlisting}[style=XMLStyle]
<SOLVERINFO>
  <I PROPERTY="EQTYPE"                VALUE="UnsteadyAdvectionDiffusion" />
  <I PROPERTY="Projection"            VALUE="Continuous"/>
  <I PROPERTY="HOMOGENEOUS"           VALUE="2D"/>
  <I PROPERTY="DiffusionAdvancement"  VALUE="Implicit"/>
  <I PROPERTY="AdvectionAdvancement"  VALUE="Explicit"/>
  <I PROPERTY="TimeIntegrationMethod" VALUE="IMEXOrder2"/>
</SOLVERINFO>
<PARAMETERS>
  <P> TimeStep      = 0.001 </P>
  <P> NumSteps      = 200   </P>
  <P> IO_CheckSteps = 200   </P>
  <P> IO_InfoSteps  = 10    </P>
  <P> wavefreq      = PI    </P>
  <P> epsilon       = 1.0   </P>
  <P> Lambda        = 1.0   </P>
  <P> HomModesY     = 10    </P>
  <P> LY            = 6.5   </P>
  <P> HomModesZ     = 6     </P>
  <P> LZ            = 2.0   </P>
</PARAMETERS>
\end{lstlisting}

By default the opeartions associated with the harmonic expansions are performed
with the Matix-Vector-Multiplication (MVM) defined inside the code. The Fast
Fourier Transofrm (FFT) can be used to speed up the operations (if the FFTW
library has been compiled in ThirdParty, see the compilation instructions). To
use the FFT routines we need just to insert a flag in the solver information as
below:

\begin{lstlisting}[style=XMLStyle]
<SOLVERINFO>
  <I PROPERTY="EQTYPE"                VALUE="UnsteadyAdvectionDiffusion" />
  <I PROPERTY="Projection"            VALUE="Continuous"/>
  <I PROPERTY="HOMOGENEOUS"           VALUE="2D"/>
  <I PROPERTY="DiffusionAdvancement"  VALUE="Implicit"/>
  <I PROPERTY="AdvectionAdvancement"  VALUE="Explicit"/>
  <I PROPERTY="TimeIntegrationMethod" VALUE="IMEXOrder2"/>
  <I PROPERTY="USEFFT"                VALUE="FFTW"/>
</SOLVERINFO>
\end{lstlisting}

The number of homogenenous modes has to be even. The Quasi-3D apporach can be
created starting from a 2D mesh and adding one homogenous expansion or starting
form a 1D mesh and adding two homogeneous expansions. Not other options
available. In case of a 1D homogeneous extension, the homogeneous direction will
be the z-axis. In case of a 2D homogeneous extension, the homogeneous directions
will be the y-axis and the z-axis.

%%% Local Variables:
%%% mode: latex
%%% TeX-master: "../user-guide"
%%% End:
