\section{Geometry}
This section defines the mesh. It specifies a list of vertices, edges (in two or
three dimensions) and faces (in three dimensions) and how they connect to create
the elemental decomposition of the domain. It also defines a list of composites
which are used in the Expansions and Conditions sections of the file to describe
the polynomial expansions and impose boundary conditions.

The GEOMETRY section is structured as \begin{lstlisting}[style=XMLStyle]
<GEOMETRY DIM="2" SPACE="2">
  <VERTEX> ...
  </VERTEX> <EDGE> ...
  </EDGE> <FACE> ...
  </FACE> <ELEMENT> ...
  </ELEMENT> <CURVED> ...
  </CURVED> <COMPOSITE> ...
  </COMPOSITE> <DOMAIN> ... </DOMAIN>
</GEOMETRY>
\end{lstlisting}
It has two attributes:
\begin{itemize}
    \item \inltt{DIM} specifies the dimension of the expansion elements.
    \item \inltt{SPACE} specifies the dimension of the space in which the
    elements exist.
\end{itemize}

These attributes allow, for example, a two-dimensional surface to be embedded in
a three-dimensional space. The \inltt{FACES} section is only present when
\inltt{DIM}=3.
The \inltt{CURVED} section is only present if curved edges or faces are present
in the mesh.

\subsection{Vertices}

Vertices have three coordinates. Each has a unique vertex ID. They are defined
in the file within VERTEX subsection as follows:
\begin{lstlisting}[style=XMLStyle] <VERTEX>
  <V ID="0"> 0.0  0.0  0.0 </V> ...
</VERTEX>
\end{lstlisting}
VERTEX subsection has three optional attributes: \inltt{XSCALE},
\inltt{YSCALE} and \inltt{ZSCALE}. They specify scaling factors to
corresponding vertex coordinates.
For example, the following snippet

\begin{lstlisting}[style=XMLStyle] 
<VERTEX XSCALE="5">
  <V ID="0"> 0.0  0.0  0.0 </V> <V ID="1"> 1.0  2.0  0.0 </V>
</VERTEX>
\end{lstlisting}

defines two vertices with coordinates $(0.0,0.0,0.0), (5.0,2.0,0.0)$. Values of
\inltt{XSCALE}, \inltt{YSCALE} and \inltt{ZSCALE} attributes can be arbitrary
analytic expressions depending on pre-defined constants, parameters defined earlier in the XML file and
mathematical operations/functions of the latter. If omitted, default scaling
factors 1.0 are assumed.

\subsection{Edges}

Edges are defined by two vertices. Each edge has a unique edge ID. They are
defined in the file with a line of the form 

\begin{lstlisting}[style=XMLStyle]
<E ID="0"> 0 1 </E>
\end{lstlisting}

\subsection{Faces}

Faces are defined by three or more edges. Each face has a unique face ID. They
are defined in the file with a line of the form

\begin{lstlisting}[style=XMLStyle]
<T ID="0"> 0 1 2 </T>
<Q ID="1"> 3 4 5 6 </Q>
\end{lstlisting}

The choice of tag specified (T or Q), and thus the number of edges specified depends on the geometry of the face (triangle or quadrilateral).

\subsection{Element}

Elements define the top-level geometric entities in the mesh. Their definition depends upon the dimension of the expansion. For two-dimensional expansions, an element is defined by a sequence of three or four edges. For three-dimensional expansions, the element is defined by a list of faces. Elements are defined in the file with a line of the form
\begin{lstlisting}[style=XMLStyle]
<T ID="0"> 0 1 2 </T>
<H ID="1"> 3 4 5 6 7 8 </H>
\end{lstlisting}
Again, the choice of tag specified depends upon the geometry of the element. The element tags are:

\begin{itemize}
    \item \inltt{S} Segment
    \item \inltt{T} Triangle
    \item \inltt{Q} Quadrilateral
    \item \inltt{A} Tetrahedron
    \item \inltt{P} Pyramid
    \item \inltt{R} Prism
    \item \inltt{H} Hexahedron
\end{itemize}


\subsection{Curved Edges and Faces}

For mesh elements with curved edges and/or curved faces, a separate entry is used to describe the control points for the curve. Each curve has a unique curve ID and is associated with a predefined edge or face. The total number of points in the curve (including end points) and their distribution is also included as attributes. The control points are listed in order, each specified by three coordinates. Curved edges are defined in the file with a line of the form
\begin{lstlisting}[style=XMLStyle]
<E ID="3" EDGEID="7" TYPE="PolyEvenlySpaced" NUMPOINTS="3">
    0.0  0.0  0.0    0.5  0.5  0.0    1.0  0.0  0.0
</E>
\end{lstlisting}

\subsection{Composites}

Composites define collections of elements, faces or edges. Each has a unique composite ID associated with it. All components of a composite entry must be of the same type. The syntax allows components to be listed individually or using ranges. Examples include
\begin{lstlisting}[style=XMLStyle]
<C ID="0"> T[0-862] </C>
<C ID="1"> E[68,69,70,71] </C>
\end{lstlisting}

\subsection{Domain}

This tag specifies composites which describe the entire problem domain. It has the form of
\begin{lstlisting}[style=XMLStyle]
<DOMAIN> C[0] </DOMAIN>
\end{lstlisting}
