\chapter{\code{class\_topology.py}} % Main chapter title

\label{Chapter7} % For referencing the chapter elsewhere, use \ref{Chapter1} 

%----------------------------------------------------------------------------------------
Here we the implementation of the topology class.

\section{\code{\_\_init\_\_}}
The initialiser for the class.

It takes the following inputs:

\begin{itemize}
\item \code{PROC\_Z}: Integer containing the value for \(R_Z\)
\item \code{PROC\_XY}: Integer containing the value for \(R_{XY}\)
\item \code{Num\_Core\_Per\_Socket}: Integer containing the number of cores per socket.
\item \code{Num\_Sock\_Per\_Node}: Integer containing the number of sockets per node.
\item \code{Scheme}: String containing the name of the numerical scheme.
\end{itemize}

It returns the following outputs:

\begin{itemize}
\item None
\end{itemize}

It sends the following to the \code{stdout}:

\begin{itemize}
\item None
\end{itemize}

Additional Output:
\begin{itemize}
\item None
\end{itemize}

%----------------------------------------------------------------------------------------

\section{\code{Print\_Hardware}}
Output the hardware created by the class to a text file.

It takes the following inputs:

\begin{itemize}
\item None
\end{itemize}

It returns the following outputs:

\begin{itemize}
\item None
\end{itemize}

It sends the following to the \code{stdout}:

\begin{itemize}
\item None
\end{itemize}

Additional Output:
\begin{itemize}
\item \code{.txt} file containing details of hardware layout in the given instance of the class.
\end{itemize}

%----------------------------------------------------------------------------------------

\section{\code{Distribute\_Elements}}
Distribute the elements.

It takes the following inputs:

\begin{itemize}
\item \code{Num\_Element\_Msg}:  Lists containing the number of elements that each core needs to communicate. 
\item \code{Num\_Elements}: List containing the number of elements assigned per core.
\end{itemize}

It returns the following outputs:

\begin{itemize}
\item None
\end{itemize}

It sends the following to the \code{stdout}:

\begin{itemize}
\item None
\end{itemize}

Additional Output:
\begin{itemize}
\item None
\end{itemize}

%----------------------------------------------------------------------------------------

\section{\code{Distribute\_Modes}}
Distribute the planes.

It takes the following inputs:

\begin{itemize}
\item \code{Num\_Modes}:  Integer containing the total number of planes.
\end{itemize}

It returns the following outputs:

\begin{itemize}
\item None
\end{itemize}

It sends the following to the \code{stdout}:

\begin{itemize}
\item None
\end{itemize}

Additional Output:
\begin{itemize}
\item None
\end{itemize}

%----------------------------------------------------------------------------------------

\section{\code{Hardware\_Constant}}
Input the FLOPs to the topology.

It takes the following inputs:

\begin{itemize}
\item \code{Num\_Constants}:  Integer containing the number of constants used.
\item \code{constants}: List of the constants.
\end{itemize}

It returns the following outputs:

\begin{itemize}
\item None
\end{itemize}

It sends the following to the \code{stdout}:

\begin{itemize}
\item None
\end{itemize}

Additional Output:
\begin{itemize}
\item None
\end{itemize}

%----------------------------------------------------------------------------------------

\section{\code{Print\_Elements}}
Produces an output file showing the element distribution.

It takes the following inputs:

\begin{itemize}
\item \code{Num\_Constants}:  Integer containing the number of constants used.
\item \code{constants}: List of the constants.
\end{itemize}

It returns the following outputs:

\begin{itemize}
\item None
\end{itemize}

It sends the following to the \code{stdout}:

\begin{itemize}
\item None
\end{itemize}

Additional Output:
\begin{itemize}
\item \code{.txt} file giving the element distribution across the cores within the given instance of the topology class. 
\end{itemize}

%----------------------------------------------------------------------------------------

\section{\code{Print\_Modes}}
Produces an output file showing the plane distribution.

It takes the following inputs:

\begin{itemize}
\item \code{Num\_Constants}:  Integer containing the number of constants used.
\item \code{constants}: List of the constants.
\end{itemize}

It returns the following outputs:

\begin{itemize}
\item None
\end{itemize}

It sends the following to the \code{stdout}:

\begin{itemize}
\item None
\end{itemize}

Additional Output:
\begin{itemize}
\item \code{.txt} file giving the plane distribution across the cores within the given instance of the topology class. 
\end{itemize}

%----------------------------------------------------------------------------------------

\section{\code{CG\_Iterations}}
Input the CG iteration data to the class instance.

It takes the following inputs:

\begin{itemize}
\item \code{Pressure}: Dictionary containing the pressure CG iteration counts for each plane number.
\item \code{Velocity\_1}: Dictionary containing the velocity 1 CG iteration counts for each plane number.
\item \code{Velocity\_2}: Dictionary containing the velocity 2 CG iteration counts for each plane number.
\item \code{Velocity\_3}: Dictionary containing the velocity 3 CG iteration counts for each plane number.
\end{itemize}

It returns the following outputs:

\begin{itemize}
\item None
\end{itemize}

It sends the following to the \code{stdout}:

\begin{itemize}
\item None
\end{itemize}

Additional Output:
\begin{itemize}
\item None
\end{itemize}

%----------------------------------------------------------------------------------------

\section{\code{Data\_Size}}
Input the value for P in order to generate the data sizes.

It takes the following inputs:

\begin{itemize}
\item \code{P}: Integer containing the degree of the basis polynomial.
\end{itemize}

It returns the following outputs:

\begin{itemize}
\item None
\end{itemize}

It sends the following to the \code{stdout}:

\begin{itemize}
\item None
\end{itemize}

Additional Output:
\begin{itemize}
\item None
\end{itemize}

%----------------------------------------------------------------------------------------
\section{\code{Input\_Communication}}
Input the hardware parameters for the given instance of the topology.

It takes the following inputs:

\begin{itemize}
\item \code{BW\_Node\_To\_Node}: Float containing the bandwidth between two nodes.
\item \code{LAT\_Node\_To\_Node}: Float containing the latency between two nodes.
\item \code{BW\_Socket\_To\_Socket}: Float containing the bandwidth between two sockets on the same node.
\item \code{LAT\_Socket\_To\_Socket}: Float containing the latency between two sockets on the same node.
\item \code{BW\_Core\_To\_Core}: Float containing the bandwidth between two sockets on the same socket.
\item \code{LAT\_Core\_To\_Core}: Float containing the latency between two sockets on the same socket.
\end{itemize}

It returns the following outputs:

\begin{itemize}
\item None
\end{itemize}

It sends the following to the \code{stdout}:

\begin{itemize}
\item None
\end{itemize}

Additional Output:
\begin{itemize}
\item None
\end{itemize}

%----------------------------------------------------------------------------------------
\section{\code{Check\_Neighbour}}
Check how two processes are related, same socket, node etc.

It takes the following inputs:

\begin{itemize}
\item \code{core\_1}: Integer containing the process number.
\item \code{core\_2}: Integer containing the process number.
\end{itemize}

It returns the following outputs:

\begin{itemize}
\item \code{1}: Value if on same socket.
\item \code{2}: Value if on same node.
\item \code{3}: Value if on different nodes.
\end{itemize}

It sends the following to the \code{stdout}:

\begin{itemize}
\item None
\end{itemize}

Additional Output:
\begin{itemize}
\item None
\end{itemize}

%----------------------------------------------------------------------------------------
\section{\code{Communication\_Pairwise\_Exchange}}
Compute the time taken to calculate the length of time taken to exchange element data using the pairwise algorithm.

It takes the following inputs:

\begin{itemize}
\item None
\end{itemize}

It returns the following outputs:

\begin{itemize}
\item \code{comm\_max}: Float of time taken to exchange element data using the pairwise algorithm.
\end{itemize}

It sends the following to the \code{stdout}:

\begin{itemize}
\item None
\end{itemize}

Additional Output:
\begin{itemize}
\item None
\end{itemize}

%----------------------------------------------------------------------------------------

\section{\code{Communication\_Allreduce}}
Compute the time taken to calculate the length of time taken to perform the Allreduce step.
It takes the following inputs:

\begin{itemize}
\item None
\end{itemize}

It returns the following outputs:

\begin{itemize}
\item \code{comm\_max}: Float of time taken to perform the Allreduce step.
\end{itemize}

It sends the following to the \code{stdout}:

\begin{itemize}
\item None
\end{itemize}

Additional Output:
\begin{itemize}
\item None
\end{itemize}

%----------------------------------------------------------------------------------------

\section{\code{Communication\_Alltoall}}
Compute the time taken to calculate the length of time taken to perform the Alltoall step.

It takes the following inputs:

\begin{itemize}
\item None
\end{itemize}

It returns the following outputs:

\begin{itemize}
\item \code{comm\_max}: Float of time taken to perform the Alltoall step.
\end{itemize}

It sends the following to the \code{stdout}:

\begin{itemize}
\item None
\end{itemize}

Additional Output:
\begin{itemize}
\item None
\end{itemize}

%----------------------------------------------------------------------------------------
